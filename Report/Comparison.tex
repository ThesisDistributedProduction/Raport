% !TeX spellcheck = en_US
\chapter{Discussion}

\section{Impact of turbine failure on the decentralized solution}
One of the main challenges of decentralized systems is to continue operation in the face of communication failure or node loss. This section address the \ref{PS:Q:Availability} problem of \cref{sec:problemStatement} which deal with the availability of the decentralized solution.

As presented in figures \ref{fig:exp:availability_kill1} to \ref{fig:exp:availability_kill15} the decentralized solution can handle the loss of turbines. A turbine is considered offline if it does not publish any new data to any other turbines within a 150 ms timespan as defined by the History QoS parameter described in \cref{sec:decen:ddsconf}.
This upper limit of 150 ms is chosen because this is the upper limit of the regulation cycle time in the current Simens system.
After the turbine has been detected as offline the remaining turbines in the wind farm will share the load of the missing turbines to keep the global power production of the park as close to the global setpoint as possible. The removal of one turbine from the windfarm is illustrated in \cref{fig:exp:availability_kill1}.

Removing several turbines from the system does not impede the regulation of the wind farm either. The remaining turbines will share the extra load between themselves and continue power production. The removal of several turbines are illustrated in figures \ref{fig:exp:availability_kill5} to \ref{fig:exp:availability_kill15}.

%The time it takes the decentralized solution to detect a turbine loss and recover power production is dependent on the following parameters:
%
%\begin{itemize}
%	\item Time of turbine loss detection, which is defined by the History QoS to 150 ms, $I$.
%	\item Time between regulation cycles, in the test cases presented in \cref{sec:res:availability} the time is set to 20 ms, $S$.
%	\item Time of calulation of a new setpoint, $C$.
%	\item Time for the turbine to regulate power production according to the new setpoint. In the turbines this process is simulated with a simple regulation that may take several regulation cycles to complete $R$.
%\end{itemize}
%
%Thus the time for recovery of the power production when a turbine failure occurs can be calculated as $T = I + S + C + R$ which translates to $150 ms + 20 ms + C + R = 170 + C + R ms$. This time can be reduced by lowering the time it takes to detect a turbine is offline which increases the likelihood of detecting a delayed turbine as offline, or by lowering the time between regulation cycles which increases the likelihood of cache reads.

Internal communication and regulation are only two parameters of availability though. To ensure that the wind farm is reachable from the outside world in the face of turbine failure, sets special requirements to the component chosen to handle external communication. A component capable of handling this specific problem has been described in \cref{cha:resourceManagement}.

We also need to address the problem of data loss when a turbine is unresponsive. A turbine has several control and measurement points which are continually logged to a local data store. Should a turbine break down the local data may be destroyed with the turbine. This again sets special requirements to the component handling data storage. In \cref{sec:databaseStorage} the requirements to the data storage component is described and MongoDB was presented as the best choice to solve the availability problem of data storage. MongoDB is capable of automatic replication of data between turbines such that data is still available should a turbine failure occur. As well as replication MongoDB provides automatic sharding of a database which enables global data to be stored across a number of nodes such that no single node contains all global data. Furthermore MongoDB supports aggregation of data across turbines to calculate aggregated values as for instance the global production of the wind farm.

Making the decentralized solution robust and able to handle the loss of turbines with regards to internal communication, regulation of power production, external communication and data storage increases the overall availability of the wind farm. 

\section{Number of turbines and the impact on regulation cycle time in the decentralized solution}
\label{sec:disc:turbinesVScycletime}
This section address the \ref{PS:Q:Performance} problem of \cref{sec:problemStatement}.

In the current Siemens system the regulation cycle time of a single Park Pilot scales linearly with the number of turbines.
The aim of the decentralized solution is to detach the regulation cycle time from the number of turbines. 
Looking at \cref{fig:exp:decen:turbines} we see that the decentralized solution seems to be independent of the number of turbines if the number of turbines is sufficiently low.
From 5 to 65 turbines the regulation cycle time is nearly constant at 20 ms.
The near constant regulation cycle time is caused by the fact that the regulation cycle in the decentralized solution is not forced to wait for data before running the regulation algorithm because data is continually shared between turbines. Adding a turbine to the decentralized solution adds an extra turbine state to factor into the setpoint calculation in the regulation algorithm. This and the added network traffic for the new turbine is the consequence of adding new turbines.

When looking at regulation cycle time of the decentralized solution we must also look at the number of cache reads.
As explained in \cref{sec:exp:performance} a cache read happens when a turbine does not provide a new turbine state package before the next regulation cycle is started.
This forces the regulation cycle to use old data read from cache.
Looking at \cref{fig:exp:decen:turbines_cache} we see that the average number of cache reads in the decentralized solution are below 2 and increasing slightly until we reach 65 turbines. From there the number of cache reads increase exponentially.

The increase in regulation cycle time and cache reads when the number of turbines reaches 65 can be explained by the fact that the network equipment of the test setup is approaching maximum throughput capacity which may cause lost or delayed network packages.
Since regulation cycle time in the decentralized system is dependent on the reception time of the oldest turbine state package as explained in \cref{sec:exp:performance} the loss or delay of network packages has a direct impact on regulation cycle time.
Similarly lost or delayed network packages increases the use of cached data. The increased regulation cycle time and cache reads are thus not a limitation of the decentralized solution but a limitation imposed by the test setup. In order to create a realistic comparison we must disregard the results that are a direct effect of the limitations of our test setup.

Disregarding the limitations imposed by the test setup we see that the regulation cycle time is constant. In terms of regulation cycle time the decentralized solution scales indefinitely. As stated above there is a small increase in regulation cycle time for every turbine added because the state of this turbine has to be taken into consideration when calculating new setpoints on all other turbines. This time addition is not visible on \cref{fig:exp:decen:turbines}. Thus the regulation cycle time will be affected by the addition of turbines but the effect is so small that it is indistinguishable by other factors. Looking at the raw test data 

Still disregarding the limitations imposed by the test setup when looking at the scale factor of the number of cache reads we see another result. The number of cache reads increases slowly with a factor of around 1 cache read for every 30 turbines added, which gives a scale factor of $1 / 30 = 0.033$.

The number of cache reads can be reduced by increasing the regulation cycle time as presented in \cref{fig:exp:decen:sleep-cache}.
Thus the factor deciding the time of the regulation cycle is the maximum number of average cache reads accepted for a single regulation cycle.

\section{Comparison of the decentralized solution and the centralized solution}
\label{sec:comp:decentralizedVScentralized}
The decentralized solution and the centralized solution is built with a very different architecture. This is reflected in the difference in the way the two solutions scale with the number of turbines. Looking at the figures for regulation cycle time in relation to the number of turbines for the decentralized and the centralized solution respectively we see that the figures reveals different trends. The test results from the centralized solution presented in \cref{fig:exp:cen:turbines} shows how adding turbines increases the regulation cycle time. From 5 to 30 turbines the regulation time of the centralized solution increase slowly. After 35 turbines the increase in regulation cycle time becomes steeper. For every 15 turbines added the regulation cycle time increases with approximately $25 ms$. The relation between regulation cycle time and the number of turbines is caused by the direct impact the number of turbines have on the regulation cycle. The Park Pilot of the centralized solution must use additional time on querying each added turbine as well as wait for their individual reply before performing the regulation algorithm. Thus the scalability of the centralized solution is linear with a factor of $25 / 15 = 1.667$.

The scalability of the decentralized solution, as described in \cref{sec:disc:turbinesVScycletime}, is small enough that it is indistinguishable from other factors in our test data and therefore we cannot calculate it. Thus the scalability of the decentralized solution is close to constant. This great improvement in scalability comes with a trade off in terms of cache reads. Adding additional turbines to the decentralized solution the number of cache reads increases with a factor of $0.033$ which is still an improvement compared to the scale factor of the centralized solution. Arguably comparing the scalability of the number of cache reads in the decentralized solution with the scalability of the regulation cycle in the centralized solution is not a viable way to compare the scalability of the two solutions, but comparing the scalability of the regulation cycle of both solutions will not be a fair comparison either since the improvements in scalability of the decentralized solution comes at the price of increased numbers of cache reads.

\section{Comparison of the decentralized solution and the current Siemens system}
This section address the \ref{PS:Q:Scalability} problem of \cref{sec:problemStatement}.
Comparing the decentralized solution and the current Siemens system directly is impossible given the differences in environment and architecture. Introducing a centralized solution is an attempt to bridge this gap. By comparing the decentralized solution to the centralized solution and measuring the improvements/demotions we can transfer these measures to the current Siemens system and an imagined decentralized implementation of the current Siemens system.

Looking at \cref{sec:comp:decentralizedVScentralized} there is a clear advantage in scalability when running a decentralized solution compared to a centralized. This is underlined by the improvement in regulation cycle time of the decentralized solution compared to the centralized solution.

Decentralizing the current Siemens system will enable improvements on other areas than regulation cycle time. By removing the centralized Park Pilots and the Wind Power Supervisor of the current Siemens solution the regulation of turbines, data storage and external communication of these must be decentralized and placed in the turbines themselves. This requires the use of software components that are able to handle regulation of turbines, data storage and external communication in a fashion such that if a turbine failure occurs other turbines can increase production to make up for the missing power production, provide access to the data collected on the failing turbine and handle external communication the failing turbine may have been handling. This increases the availability of the wind farm compared to the current Siemens system.

\begin{itemize}
	\item Test parameters: Our system vs Siemens system?
	\item Redundancy, up time, scalability...
	\item Solved problems (i.g. Single point of failure)
\end{itemize}