% !TeX spellcheck = en_US
\chapter{Discussion}

\section{Impact of turbine failures}
One of the main challenges of decentralized systems is to continue normal operation in the face of communication failure or node loss. This section addresses the \ref{PS:Q:Availability} problem of \cref{sec:problemStatement} which deal with the availability of the decentralized solution.

As presented in

\section{Number of turbines and the impact on regulation cycle time in the decentralized solution}
\label{sec:disc:turbinesVScycletime}
This section addresses the \ref{PS:Q:Performance} problem of \cref{sec:problemStatement}. In the current Siemens system the regulation cycle time of a single Park Pilot scales linearly with the number of turbines.
The aim of the decentralized solution is to detach the regulation cycle time from being dependent on the number of turbines. 
Looking at \cref{fig:exp:decen:turbines} in \cref{chap:results} we see that the decentralized solution is almost independent on the number of turbines.
From 5 to 65 turbines the regulation cycle time is nearly constant on $20 ms$ with very little variation in the dataset and moderate extreme values.
The regulation cycle time is caused by the fact that the regulation cycle in the decentralized system is not forced to wait for data before running the regulation algorithm because data is continually shared between turbines.

When looking at regulation cycle time of the decentralized solution the number cache reads.
As explained in \cref{sec:exp:performance} a cache read happens when a turbine does not provide a new turbine state package before the next regulation cycle is started.
This forces the regulation cycle to use old data read from cache.
Looking at \cref{fig:exp:decen:turbines_cache} we see that the average number of cache reads in the decentralized solution are below 5 and increasing slightly until we reach 65 turbines. From there the number of cache reads increase exponentially.

The increase in regulation cycle time and cache reads when the number of turbines reaches 65 can be explained by the fact that the network equipment of the test setup is approaching maximum throughput capacity which may cause lost or delayed network packages.
Since regulation cycle time in the decentralized system is dependent on the reception time of the oldest turbine state package as explained in \cref{sec:exp:performance} the loss or delay of network packages has a direct impact on regulation cycle time.
Similarly lost or delayed network packages increases the use of cached data. The increased regulation cycle time and cache reads are thus not a limitation of the decentralized solution but a limitation imposed by the test setup.

Disregarding the limitations imposed by the test setup we see that the regulation cycle time is nearly constant while the number of cache reads increases slowly with a factor of around 1 cache read for every 30 turbines added, which gives a scale factor of $1 / 30 = 0.033$.

The number of cache reads can be reduced by increasing the regulation cycle time as presented in \cref{fig:exp:decen:sleep-cache}.
Thus the factor deciding the time of the regulation cycle is the maximum number of average cache reads accepted for a single regulation cycle.

\section{Comparison of decentralized solution and centralized solution}
The decentralized solution and the centralized solution is built with a very different architecture. This is reflected in the difference in the way the two solutions scale with the number of turbines. Looking at the figures for regulation cycle time in relation to the number of turbines for the decentralized and the centralized solution respectively we see that the figures reveals very different trends. The test results from the centralized solution presented in \cref{fig:exp:cen:turbines} shows how adding turbines increases the regulation cycle time. From 5 to 30 turbines the regulation time of the centralized solution increase slowly. After 35 turbines the increase in regulation cycle time becomes steeper. For every 15 turbines added the regulation cycle time increases with approximately $25 ms$. The relation between regulation cycle time and the number of turbines is caused by the direct impact the number of turbines have on the regulation cycle. The Park Pilot of the centralized solution must use additional time on querying each added turbine as well as wait for their individual reply before performing the regulation algorithm. Thus the scalability of the centralized solution is linear with a factor of $25 / 15 = 1.667$.

The scalability of the decentralized solution is, as described in \cref{sec:disc:turbinesVScycletime}, does also scale linearly but the scale factor is much smaller at $0.033$.

Comparing the the scalability of the centralized solution to the decentralized solution we see that the decentralized solution scales $1.667 / 0.033 = 50.51$ times better than the centralized solution.

\section{Comparison of decentralized and current Siemens system}
Comparing the decentralized and the current Siemens system directly is impossible given the differences in environment and architecture. Introducing a centralized solution is an attempt to bridge this gap. By comparing the decentralized solution to the centralized solution and measuring the improvements/demotions we can transfer these measures to the current Siemens system and an imagined decentralized implementation of the entire system. 

\begin{itemize}
	\item Test parameters: Our system vs Siemens system?
	\item Redundancy, up time, scalability...
	\item Solved problems (i.g. Single point of failure)
\end{itemize}