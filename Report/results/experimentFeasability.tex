
\section{\ref{PS:Q:Feasibility}}

\textit{How can we best re-implement the current Siemens system (\cref{sec:SiemensCase}) as a system where the Wind Power Supervisor and the Park Pilots are decentralized?}\newline\newline

\noindent To answer whether or not the current Siemens system can be fully decentralized, the discussion section of this experiment also contains a discussion with regards to the components analyzed in \cref{cha:stateOfTheArt}: State of the Art.

\subsection{Experiments}

The \ref{PS:Q:Feasibility} problem address the possibility of reimplementing the current Siement system as a decentralized system. The following experiment aims to investigate if the decentralized solution is able to regulate the power production of each turbine in order to maintain the global power production goal.

The experiment has the following procedure:
\begin{enumerate}
	\item Start the system with 10 turbines.
	\item Start the graphical interface described in \cref{sec:graphicalInterface}.
	\item Observe that the global setpoint and the global production line on the graphical interface match.
\end{enumerate}

\subsection{Results}
\label{sec:exp:feasibility}
\begin{figure} [!h]
	\centering
	\includegraphics[width=\resultsFigureWidthScale\textwidth]{gui.png} 
	\captionsetup{format=plain,font=footnotesize,labelfont={bf,defaultCapFont},labelsep=quad,singlelinecheck=no}
	\caption[Graphical interface running 10 turbines]{
		\label{fig:graphicalInterface} 
		\footnotesize{%
			Graphical interface running 10 turbines.
		}
	}
\end{figure}

Presented in \cref{fig:graphicalInterface} is a screenshot of the graphical interface made with DDS Blockset for Simulink (described in section \cref{sec:graphicalInterface}). The x-axis show Time in seconds and the y-axis show kW production. The screenshot is taken while the decentralized solution runs 10 turbines.
The global setpoint for power production is illustrated by the red line and is showing a value of just below 5000 kW. The global power production is illustrated by the black line also just below 5000 kW.
The blue line illustrates the maximum available power production for the wind farm which, during this experiment, was above 10000 kW. The lines at the bottom shows values of around 500 kW and they each represents the current power production of each turbine.

\subsection{Discussion}\label{feas:discussion}
The experiment performed to prove whether or not it is feasible to create a decentralized solution that is able to do power regulation, consists of a simple run of the prototype of the decentralized solution. The global setpoint and the global production line match and thus we conclude that we have a working prototype, that can regulate according to the global setpoint. Each turbine is expected to produce $globalSetpoint/nTurbines=5000kW/10=500kW$(see \cref{sec:calculateSetpoints} for detailed description of the regulation algorithm), which they do. The slight deviation (the deviation that makes each turbine produce close to 500kW and not exactly 500kW) is due to the regulation algorithm being inaccurate. For the purpose of this experiment, the fact that the regulation algorithm is inaccurate is acceptable since the regulation algorithm is black box and not the focus of this thesis.

As such it is hard to deduce from a screenshot that the decentralized solution is actually able to perform power regulation, but this experiment in correlation with the rest of the experiments performed in this chapter will show that the prototype is functioning.

In order to decentralize fully decentralize the current Siemens system, decentralization of the Wind Power Supervisor must also be covered. The purpose of the Wind Park Supervisor is to log data from every turbine within the farm and handle external communication. Thus the Wind Park Supervisor consists of two primary features: Data storage and aggregation and external communication. 

Decentralizing the data storage and aggregation onto the turbines requires a horizontally scalable distributed database, which can handle data aggregation. Distributing the database horizontally is important in order to handle datasets larger than the physical storage of one turbine and data aggregation is needed for the case where an external request requires data to be aggregated. To handle this MongoDB~\cite{mongodb} has been identified as the optimal component (as concluded in \cref{sec:databaseStorage} Data Storage). MongoDB supports Sharding for automatic horizontal partitioning of datasets and data aggregation. MongoDB has been chosen based on other parameters related to the problems \cref{PS:Q:Availability} and \cref{PS:Q:Performance}, however to answer the answer the \cref{PS:Q:Feasibility} problem, this functionality is sufficient. 

Handling external communication decentralized means, BLABLA has been identified as the optimal component.

Thus we conclude that decentralizing the entire current Siemens system is theoretically feasible, based on the analysis of state of the art technologies made. 
\clearpage