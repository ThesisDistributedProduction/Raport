\chapter{Abstract}
Wind energy is widely recognized as one of the most cost efficient renewable energy sources. 
%Because of this wind farms are increasing in size and power production. 
%Because of this wind farm size and power production is increasing.
Because of this wind farm size and power production steadily increases.
%, in order to accommodate the rising need for energy from renewable sources. 
%The number of turbines in a single wind farm can reach more than 500.
Control of the increasing number of turbines in wind farms is becoming problematic.
%
The traditional hierarchical control approach with central control points responsible for regulation of turbine power production does not scale well with the number of turbines, and introduce single points of failure.
%
This thesis aims to take a different approach to control of wind farms.
%
In place of the hierarchical control approach the aim is to decentralize the control of turbines.
%
%This is done by letting each turbine regulate it's own power production in accordance with the power production of all other turbines in the wind farm.
This is achieved by letting the turbines regulate power production themselves while cooperating to reach the power production setpoint of the wind farm.
%In place of the hierarchical control approach the aim is to decentralize the control of turbines in a wind farm by letting each turbine control it's own power production in accordance with the power production of all other turbines in the wind farm.
%In place of the hierarchical control approach this thesis aim to let every turbine control itself in accordance with the power production of other turbines, in effect decentralizing the control of the wind farm.
Decentralizing the control of turbines will increase the scalability of the wind farm as increasing the number of turbines will also increase the number of resources available to perform regulation of power production.
%
As well as increased scalability, decentralizing wind farm control will increase availability, as there no longer exist a single point in the farm that every turbine is dependent on.

\chapter{Resumé}
Vindenergi er bredt anerkendt som en af de mest omkostningseffektive vedvarende energikilder.
Derfor stiger vindmølleparkers størrelse og elproduktion støt.
Kontrol af det stigende antal møller i en vindmøllepark er problematisk.
Den traditionelle hierarkiske kontrol tilgang med centrale kontrolpunkter, der har ansvar for at regulere turbinernes elproduktion skalerer ikke godt med antallet af turbiner, og introducerer single points of failure.
Dette speciale har til formål at undersøge en alternativ tilgang til kontrol af vindmølleparker.
I stedet for den hierarkiske kontrol tilgang er målet at decentralisere kontrollen af turbiner.
Dette opnås ved at lade turbinerne regulere deres elproduktion selv, mens de samarbejder for at nå vindmølleparkens sætpunkt for elproduktion.
Decentraliseringen af kontrollen med turbiner vil øge vindmølleparkens skalerbarhed, øges antallet af turbiner vil antallet af ressourcer der kan udføre regulering af elproduktion også øges.
Samtidig med at skalerbarheden øges, vil en decentralisering af kontrollen af turbiner i en vindmøllepark øge vindmølleparkens tilgængelighed, da der ikke længere eksisterer et centralt punkt alle turbiner er afhængige af.