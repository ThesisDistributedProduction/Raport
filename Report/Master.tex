\documentclass[a4paper,11pt, oldfontcommands]{memoir}
%\documentclass[b5paper,10pt,twoside]{memoir}
% Ved a4 format skal nedenst�ende nok justeres en del
\setlrmarginsandblock{3.5cm}{2.5cm}{*}
\setulmarginsandblock{3cm}{*}{1.2}

\usepackage[retainorgcmds]{IEEEtrantools}
\usepackage{graphicx}
\usepackage{pstricks,pst-node,pst-text,pst-3d}
\usepackage{marginnote}
\usepackage{array,booktabs,longtable}
\usepackage{url}
\usepackage{color}
%\definecolor{webblue}{rgb}{0.0,0.05,0.45}
\definecolor{MyDarkBlue}{rgb}{0,0.08,0.45}
%\definecolor{webred}{rgb}{0.75,0,0}

\usepackage[T1]{fontenc}
\checkandfixthelayout
% ops�tning af pagestyles
\makepagestyle{book} % laver en ny tom pagestyle
\makeevenhead{book}{}{\small\sffamily\leftmark}{}
\makeoddhead{book}{}{\small\sffamily\rightmark}{}
\makeevenfoot{book}{\small\sffamily\thepage}{}{}
\makeoddfoot{book}{}{}{\small\sffamily\thepage}
\makeatletter

\makepsmarks{book}{%
\renewcommand\chaptermark[1]{%
\markboth{%
\ifnum \value{secnumdepth} > 1
\if@mainmatter % indenfor frontmatter er der intet kapitel nummer
\@chapapp\ \thechapter. \ % \@chapapp er lidt dum, se nedenfor
\fi
\fi
##1}{}}%
\renewcommand\tocmark{\markboth{\contentsname}{\contentsname}}%
\renewcommand\lofmark{\markboth{\listfigurename}{\listfigurename}}%
\renewcommand\lotmark{\markboth{\listtablename}{\listtablename}}%
\renewcommand\bibmark{\markboth{\bibname}{\bibname}}%
\renewcommand\indexmark{\markboth{\indexname}{\indexname}}%
\renewcommand\sectionmark[1]{\markright{##1}}%
\renewcommand\subsectionmark[1]{\markright{##1}}%
\renewcommand\subsubsectionmark[1]{\markright{##1}}%
}
\makeatother

\pagestyle{book}
% laver om p� plain stilen
\copypagestyle{plain}{book}
\makeevenhead{plain}{}{}{}
\makeoddhead{plain}{}{}{}

%**************** Headlines ****************************************************
% konfiguration af kapitel titel font samt fonte til sections
\renewcommand\chapnamefont{\Huge\bfseries\sffamily}
\renewcommand\chapnumfont{\chapnamefont}
%\renewcommand\chaptitlefont{\color{webred}\Huge\usefont{OT1}{phv}{bc}{n}\selectfont\raggedright}
\renewcommand\chaptitlefont{\Huge\usefont{OT1}{phv}{bc}{n}\selectfont\raggedright}
%\setsubsecheadstyle{\color{MyDarkBlue}\large\bfseries\sffamily\raggedright}
\setsubsecheadstyle{\large\bfseries\sffamily\raggedright}
%\setsubsubsecheadstyle{\color{webred}\Huge\usefont{OT1}{phv}{bc}{n}\selectfont\raggedright}
\setsubsubsecheadstyle{\Huge\usefont{OT1}{phv}{bc}{n}\selectfont\raggedright}
% Rule under headline.. (Underline section headlines med de 4 linier herunder) 
%\newcommand{\ruledsec}[1]{%
%\Large\usefont{OT1}{phv}{b}{n}\selectfont\raggedright #1 %\color{webred}\rule[15pt]{\textwidth}{1.0pt}} 
%\setsecheadstyle{\ruledsec} %ud-kommenter linie herunder
\setsecheadstyle{\Large\usefont{OT1}{phv}{b}{n}\selectfont\raggedright}

\usepackage{titlesec}
\usepackage{titletoc}
% Afstande mellem overskrifter og tekst
\titleformat{\subsection}
%{\color{MyDarkBlue}\usefont{OT1}{phv}{b}{n}\selectfont}{\thesubsection}{1em}{}
{\usefont{OT1}{phv}{b}{n}\selectfont}{\thesubsection}{1em}{}
\titleformat{\subsubsection}
%{\color{MyDarkBlue}\usefont{OT1}{phv}{b}{n}\selectfont}{\thesubsubsection}{1em}{}
{\color{black}\usefont{OT1}{phv}{b}{n}\selectfont}{\thesubsubsection}{1em}{}
\titlespacing{\subsection}{0pt}{15pt}{10pt}
\titlespacing{\subsubsection}{15pt}{15pt}{10pt}
\titlespacing{\section}{15pt}{35pt}{10pt}
%***********  section number in margin ****************************************
\makeatletter
\def\@seccntformat#1{\@ifundefined{#1@cntformat}%
{\csname the#1\endcsname\quad}% default
{\csname #1@cntformat\endcsname}% individual control
}
%\def\section@cntformat{\color{webred}\protect\makebox[0pt][r]{\thesection.\quad}}
\makeatother


% s�g for at hvis en \section el.lign. flyttes til en ny side
% da skal siden f�r ikke str�kkes
\raggedbottomsectiontrue
% justering af afsnitsnummerering og ToC dybde
\setsecnumdepth{subsubsection} % til og med
\maxsecnumdepth{subsubsection} % underligt koncept
%\settocdepth{subsection} %til og med
% nogle pakker man ofte anvender
\usepackage[ansinew]{inputenc} % eller ansinew hvis Windows
%*******************************************************************************
\usepackage[T1]{fontenc}
\usepackage{amsmath,amssymb}
\usepackage{mathtools}
\usepackage{graphicx}
\usepackage{fix-cm,fixltx2e}
\usepackage{soul}

\DeclareRobustCommand{\SetFourierSpace}{%
\fontdimen2\font=1.13\fontdimen2\font}
\sodef\an{}{0.13em}{0em}{0em} \sodef\ann{}{0.13em}{0.5em}{0em}
%******************************************************************************
\let\footruleskip\relax % for compatibility of memoir and fancyhdr
%% Use fancy chapter headers, with Jos Dingjan's modifications,
%% plus my own tweaks. This style is not part of teTeX,
%% so we are using a local (and renamed) copy. Reverted to original!
% \usepackage[Lenny]{fncychapleo}
\usepackage[Lenny]{fncychap}
\usepackage{fancyhdr}

% Setup af fncychap Lanny chapter header
\ChNameVar{\LARGE\usefont{OT1}{phv}{m}{n}\selectfont\raggedright}
\ChNumVar{\fontsize{75}{10}\usefont{OT1}{cmr}{m}{n}\selectfont\raggedright}
\ChTitleVar{\Huge\usefont{OT1}{phv}{bc}{n}\selectfont\raggedright}

%*******************************************************************************

\makeatletter
\newcommand\figcaption{\def\@captype{figure}\caption}
\newcommand\tabcaption{\def\@captype{table}\caption}
\makeatother

\newenvironment{narrow}[2]{%
\begin{list}{}{%
\setlength{\topsep}{0pt}%
\setlength{\leftmargin}{#1}%
\setlength{\rightmargin}{#2}%
\setlength{\listparindent}{\parindent}%
\setlength{\itemindent}{\parindent}%
\setlength{\parsep}{\parskip}}%
\item[]}{\end{list}}

\newlength{\marginwidth}
\setlength{\marginwidth}{2.0\oddsidemargin} %bredden af billeder ud i margin'en

% Picture Handling
\usepackage{subfig}
\usepackage[leftcaption]{sidecap}
%\usepackage{varioref} %For smarte referencer med \vref istedet for \ref
\usepackage[hidelinks]{hyperref}
\usepackage{cleveref}
\usepackage{calc}% auto udregn

\usepackage{tikz}
\usepackage{pgf-umlsd}
\usetikzlibrary{arrows}

\graphicspath{{figures/}}
\sidecaptionvpos{figure}{t} %sidecaption aligned med toppen af billede
\usepackage{caption}
%\DeclareCaptionFont{red}{\color{webred}}
\DeclareCaptionFont{defaultCapFont}{\color{black}}
\captionsetup{singlelinecheck=false,font=footnotesize,labelfont={bf,defaultCapFont},format=hang}

\DeclareCaptionFormat{llap}{\llap{#1#2}#3\par}
\strictpagechecktrue
\makeatletter
\DeclareRobustCommand*{\bfseries}{%
  \not@math@alphabet\bfseries\mathbf
  \fontseries\bfdefault\selectfont
  \boldmath
}
\makeatother

% andre pakker og konfigurationer
\usepackage{makeidx}
\usepackage{mflogo} 
\makeindex % hvis man anvender s�dan en

%\renewcommand*{\cftchaptername}{Chapter\space}
\renewcommand*{\cftchaptername}{}
\renewcommand*{\cftfigurename}{Fig.\space}
\renewcommand{\contentsname}{Table of Contents}
\renewcommand*\abstractname{Summary}

%\includeonly{introduction} %Hvis man ikke vil kompilere det hele hver gang
\begin{document}
\frontmatter

\chapter{Abstract}
Wind energy is widely recognized as one of the most cost efficient renewable energy sources. 
%Because of this wind farms are increasing in size and power production. 
%Because of this wind farm size and power production is increasing.
Because of this wind farm size and power production steadily increases.
%, in order to accommodate the rising need for energy from renewable sources. 
%The number of turbines in a single wind farm can reach more than 500.
Control of the increasing number of turbines in wind farms is becoming problematic.
%
The traditional hierarchical control approach with central control points responsible for regulation of turbine power production does not scale well with the number of turbines, and introduce single points of failure.
%
This thesis aims to take a new approach to control of wind farms.
%
In place of the hierarchical control approach the aim is to decentralize the control of turbines.
%
%This is done by letting each turbine regulate it's own power production in accordance with the power production of all other turbines in the wind farm.
This is achieved by letting the turbines regulate power production themselves while cooperating to reach the power production goal of the wind farm.
%In place of the hierarchical control approach the aim is to decentralize the control of turbines in a wind farm by letting each turbine control it's own power production in accordance with the power production of all other turbines in the wind farm.
%In place of the hierarchical control approach this thesis aim to let every turbine control itself in accordance with the power production of other turbines, in effect decentralizing the control of the wind farm.
Decentralizing the control of turbines will increase the scalability of the wind farm as increasing the number of turbines will also increase the number of resources available to perform regulation of power production.
%
As well as increased scalability, decentralizing wind farm control will increase availability, as there no longer exist a single point in the farm that every turbine is dependent on.

\chapter{Resumé}

\cleartorecto
\tableofcontents

\mainmatter
\chapter{Introduction}

\section{Thesis motivation}
Today windmills in a windmill farm are connected to a single server that aggregates data, perform calculations, store data and handle communication with the outside world. 
These servers do not scale well with the size of the windmill farm, and they are a single point of failure. 
Therefor Siemens wishes to remove the servers by utilizing free capacity of the computers already residing in every windmill. 
Currently there is some limited redundancy in data and availability but this could be greatly improved by distributing data and communication to the windmills. 
Ease of access must be maintained even though computation and data is distributed. 
This means routing traffic to a windmill with free capacity through. 
This means routing traffic to a windmill with free capacity through  a single interface.

\section{Thesis aim}

\begin{itemize}
	\item How do we distribute a database and computation across a production environment in the best possible way?
	\item How do we define and measure performance?
	\item Can it provide data redundancy and outperform current systems?
	

\end{itemize}

We aim to investigate the possibility of making a framework and associated tools for developing a distributed system. 
This framework must be able to handle computation distributed on several nodes, communication between those nodes and distribution of data. 
The communication can be built on top of existing standards as for instance DDS. Data distribution can be built using existing systems like MongoDB. 
Distribution of computation tasks is the main research area and will be the focus of this thesis.

In order to achieve distributed computation on several nodes the framework must be able to perform load balancing and control the distribution of tasks on the nodes in the system. 
Furthermore the framework must have a single interface for control of, and interaction with, all the nodes.  
The goal is to create a test system, that can distribute and perform tasks but also to be able to plan ahead of time and know if there is available computation time.

The case from Siemens Windpower is an example of a production environment where the framework could be utilized. 
Our goal is not to make a framework that is only  specific for this case but to make a general framework for this and similar cases.

\section{Approach}

\section{Outline}
The remainder of this thesis is organized into the following chapters...

\section{Audience}
This thesis is aimed at an audience with a basic knowledge of...

%\subsection{Siemens case}

Siemens Wind Power is among the leading windmill manufacturers in the world. 
Siemens builds wind farms of different sizes ranging form single mills to well above one hundred wind mills \cite{simensOffShoreProjects, simensOnShoreProjects}.

Siemens has a need for their system to scale better and provide increased redundancy.
In the current setup (see \cref{fig:currentSiemensSetup}) the Park Monitor is a single point of failure.
Siemens has a vision of removing this component and make it into a distributed system, distributed among the windmills (see \cref{fig:futureSiemensSetup}).

The system needs to minimum handle 50 control points and 200 measurement points.

\begin{figure}
	\centering
	\includegraphics[width=0.7\textwidth,natwidth=610,natheight=642]{SystemOverviews.png} 
	\captionsetup{format=plain,font=footnotesize,labelfont={bf,red},labelsep=quad,singlelinecheck=no}
	\caption[Illustrates the current Siemens windmill farm setup]{
		\label{fig:currentSiemensSetup} 
		\footnotesize{%
			This figure illustrates the current Siemens windmill farm setup.
		}
	}
\end{figure}

\begin{figure}
	\centering
	\includegraphics[width=0.7\textwidth,natwidth=610,natheight=642]{SystemOverviewsFuture.png} 
	\captionsetup{format=plain,font=footnotesize,labelfont={bf,red},labelsep=quad,singlelinecheck=no}
	\caption[Illustrates the future Siemens windmill farm setup]{
		\label{fig:futureSiemensSetup} 
		\footnotesize{%
			This figure illustrates the future Siemens windmill farm setup.
		}
	}
\end{figure}

%\include{dedication}
%\chapter{State of the art}
Wind energy is by far one of the most popular green energy forms.
It is predicted that by the end of 2014 the worlds collected wind energy production will be near 360GW\cite{worldwidewindcapacity}.
Despite the focus on and popularity of wind energy there is still a lot of engineering challenges to be overcome.
The following sections will describe state of the art research with respect to control and communication within the wind energy sector.

\section{Control}
One area that is still open for optimization is the control of turbines both as a single unit but also as a farm. 
Calculating the perfect position for each turbine such that the global power production setpoint is reached is a very hard problem.
The algorithms must take a lot of parameters into account as for instance noise produced by the wings, certain speeds where the wear and tear of the turbine is high and the position of other turbines so the wake created by one turbine does not reduce the performance of other turbines unnecessarily.
Furthermore every turbines maximum power production and current power production must also be taken into account.
Several approaches has been tested as described in the following subsections. %TODO: Rewrite,

\subsection{Hierarchical/centralised control}
The hierarchical/centralised approach uses local control on the turbine level and global control on the wind farm level\cite{CentralisedPowerControlOfWindFarm, HeirarchicalWindFarmControl}.

Setpoints for the global output of the wind farm are received by the controller on the wind farm level.
So is the output for each turbine and the maximum available output for each turbine.
The global controller calculates setpoints for each turbine based on the global setpoint and each turbines current and possible output.

The controllers on turbine level is responsible for reaching the setpoint calculated by the global controller as well as making each turbine reach the setpoint in the most optimal manner(gearing, avoid ice over, avoid oscillation).

The hierarchical/centralized approach is the current approach used in the Siemens case.

\subsection{Decentralized feed-forward control}
The decentralized feed-forward approach\cite{DecentralisedFeedforwardControlOfWindFarms} takes advantage of the fact that turbines are placed in a farm by letting upwind turbines feed wind data to downwind turbines. 
This allows downwind turbines to make adjustments to their production in order to exploit the coming wind in the best way.
Furthermore a restricted communication model is used allowing turbines only to communicate with their neighbors.

Using this decentralized feed-forward approach to control a wind farm can help even out the output of the farm since downwind turbines has additional information regarding wind speed to come to act upon. If upwind turbines power production is also a part of the feed-forward package downwind turbines may also be able to regulate overall wind farm production by evening out spikes from upwind turbines.
In addition by only communicating with neighboring turbines in order to achieve improvements in output the need for a centralized node is alleviated.

\subsection{Game theory control}
A new approach to control of wind farms is to utilize game theory\cite{AModelFreeApproachToWindFarmControl}.
The turbines in a farm must cooperate to reach the desired goal of a chosen output current.
The game theory approach use an iterative learning algorithm that converges against the optimal output after n iterations.
According to the above referenced article improvements on up to 25\% is possible compared to other algorithms currently in use.
%\chapter{State of the art}\label{cha:stateOfTheArt}

The problem statement (\cref{sec:problemStatement}) presents four questions for this project. The \ref{PS:Q:Feasibility} question being the primary question, from which the other three spawns. 

To answer the \ref{PS:Q:Feasibility} question we have to identify the key services performed by the Wind Power Supervisor and the Park Pilots. These services must be provided by the new decentralized system such that removing the central control points, the Wind Power Supervisor and the Park Pilots, will not diminish the new decentralized systems function compared to the current system as described by the Siemens Case in \cref{sec:SiemensCase}. The following key services has been identified:

\begin{enumerate}[label=\textbf{\alph*.}, ref=\textit{\alph*}]
\item \label{Analysis:need:a} \textbf{Data aggregation} \\
	It must be possible to aggregate data across the entire wind farm in order to enable calculation of global parameters such as global power production, global max power production etc. 

\item \label{Analysis:need:b} \textbf{Data storage} \\
	The individual turbines must be able to store data about wind speeds, individual power production as well as the state of wear on different parts of the turbine. Several data points are measured every second. As well as individual data, global data must be stored as well. Without a central storage point the global data must be distributed across the entire wind farm such that it is accessible by any turbine handling an external connection which may request such global data.

\item \label{Analysis:need:c} \textbf{External communication} \\
	The wind farm must be able to handle communication with the external world. Currently this is handled by both the Wind Power Supervisor and the Park Pilots depending on which data is relevant to the initiator of the external connection. The external communication must be handled by the turbines of the wind farm such that it is transparent to the initiator which node is actually performing the services requested.

\item \label{Analysis:need:d} \textbf{Park regulation} \\
	The wind farm must regulate the power production towards a global power production setpoint. Without a central controller to calculate the individual setpoints for all turbines these calculations must be done by each individual turbine.
	
\end{enumerate}

For the wind farm to be able to handle the needs above and act as a single unit, we are forced to look at decentralized software components. 

The following chapters describes, analyses and discusses solutions to the needs above. The chapters are divided based on the components we see as solutions to \ref{Analysis:need:a}-\ref{Analysis:need:d}. As such this analysis contains the following chapters:

\begin{enumerate}[label=2.\arabic*]
\item{\textbf{Database storage.}} The obvious solution for need \ref{Analysis:need:a} and \ref{Analysis:need:b}.
\item{\textbf{Resource management.}} In the decentralized system, the turbines needs to be able to handle need~\ref{Analysis:need:c}. Currently Siemens windmill farms can have up to 100 communication channels open at a time. In order to avoid a single mill having to handle 100 communication channels, resource management has to be introduced to the system.
\item{\textbf{Distributed computing.}} For the turbines to be able to perform park regulations, communication between the turbines is essential, since park regulation demands knowledge about every turbine in the farm. Distributed computing is introduced to the system as a way to keep a global state as a solution to need~\ref{Analysis:need:d}. This chapter also addresses question~\ref{PS:Q:Scalability} of the problem statement.
\end{enumerate}

Question \ref{PS:Q:Availability} and \ref{PS:Q:Performance} of the problem statement are addressed by all of the chapters mentioned above.

%Problem -> Alternativer -> Sammenligning -> Endeligt valg
%
%
%Opdeling vs. samme emne
%\begin{itemize}
%\item Vidensudveksling er vigtig
%\item  Opsplitning med undervisning
%\end{itemize}
%
%
%noter:
%pup/sub pattern vs DSM (Distributed shared memory)
%ved multicast operationer fejler mange pup/sub systemer da de bruger almindelige unicast protokoller og derfor sender samme pakke flere gang til forskellige modtagere istedet for at multicast pakken en gang til all. (speciale rapport kilde: Unders\o gelse af Distributed Shared Memory performance og anvendelse ...)
%
%


\section{Database storage}
\label{sec:databaseStorage}
Data storage is important in a system like a wind farm.
A lot of data must be persisted like weather data, health data of the different parts of every turbine and production data.
These records are important both for immediate use to view the current state of the system but also for review in the future for instance to predict weather trends or replace worn down parts of turbine before they break completely.

Currently data is aggregated from each turbine and stored on a central node.
This node will over time aggregate hundreds of gigabytes of information.
The data on the node is secured by backup but it is still a single point of failure.
Take out the data storage node or the communication to it and a lot of information will be lost.

By distributing the data of the system between all the connected nodes we achieve better redundancy because the data is present on many different nodes.
Should a node become unavailable another node can communicate the same data in effect strengthening the availability of the system.

This chapter contains a description of a number of relevant data storage technologies and a discussion of which technology is the best suited for a system like the Siemens case presented in \cref{sec:SiemensCase}.

Storage technologies will be compared on a set of parameters that are relevant to the case presented in \cref{sec:SiemensCase}. The parameters are presented below in prioritized order:

\begin{enumerate}
\item \textbf{Scalability} \\
The storage technology must be able to scale horizontally in order to allow a varying number of nodes.

\item \textbf{Availability} \\
The data in the system must be available for processing at any time.

\item \textbf{Replication} \\
The data in the system must be replicated between nodes in order to avoid data loss should one node be damaged.
\item \textbf{Failover} \\
The data storage technology must be able to seamlessly switch from a damaged node to a working node if a failure occurs.

\item \textbf{Sharding} \\
The data storage technology must support automatic partitioning of datasets that are larger than the physical storage on one node.

\item \textbf{Aggregation} \\
The data storage technology must support aggregation of data.
\end{enumerate}

\subsection{Relational storage, SQL}
\label{sec:sql}
The traditional way of storing data is in a Relational Database Management System(RDBMS).
These databases rely on a schema to arrange data in tables and their relations.
Using SQL it is easy to query data and to do aggregate operations.
RDBMSs support ACID transactions which ensures operations in the database are processed reliably.

%A shortcoming of the RDBMSs is the problem with object-relational mapping also known as the Impedance Mismatch Problem\cite{Fowler:IntroNoSQL, Neward:TheVietnamOfComputerScience}.
%The relational structure of the RDBMSs does not map well to the object-oriented structure the most popular programming languages encourage.
%Often an object is an aggregate of a number of attributes.
%In the context of the object-oriented program the object is seen as one entity.
%In the context of the RDBMS the attributes of the object-oriented object is often scattered between multiple tables in the database to ensure consistency and avoid duplicate data.
%This mismatch between object representation and relational representation can cause both performance problems, the JOIN operation in SQL is very costly, as well as considerable development time spent mapping one structure to the other.
%The performance problem multiplies in a distributed database if the RDBMS must do JOIN operations across the network in order to aggregate data.

A problem with a traditional RDBMS is that they are designed for vertical scaling~\cite{Atzeni:TheRelationalModelIsDead}. If a traditional RDBMS has problems handling data the solution is to add a bigger harddrive or invest in a faster CPU. This makes sense in a world were hardware is very expensive like it was when the traditional RDBMSs saw the light of day~\cite{Stonebraker:TheEndOfAnArchitecturalEra}. Today horizontal scaling is preferred. If a system has a problem with the data load add another machine or add five others if that is what it takes. Since horizontal scalability is a very important feature of the data storage system the traditional RDBMS is not a viable option.

\subsection{Schema-less storage, NoSQL}
\label{sec:nosql}
Since 2009 the schema-less storage methods have become increasingly popular.
Relational databases could no longer keep up with the task of storing and querying big data.
A new breed of schema-less storage systems became popular because they could handle some of the problems big data caused for the relational storage systems.
This new breed of databases are designed for horizontal scalability and without strict schemas allowing a more flexible data model. 
They are capable of handling large scale amounts of data both for data storage but also for analysis or batch operations.
The downside however is the lack of the ACID properties which results in decreased consistency and the lack of transactions. %TODO: Dårligt formuleret omskriv!

The schema-less databases can be divided roughly into four categories~\cite{Fowler:IntroNoSQL, Moniruzzaman:NoSQLDatabaseNewEraOfDatabasesForBigDataAnalysis}:

\begin{itemize}
	\item \textbf{Document databases} \\
	The document databases are primarily used to store semi structured data in the form of documents. The data is stored in attribute name-value pairs, where the attributes may vary between rows.
	
	\item \textbf{Key-value databases} \\
	The key-value databases are primarily used for fast lookup of data based on a key. The data is stored in key-value pairs.
	
	\item \textbf{Column-family databases} \\
	The column-family stores are primarily used for distributed data storage, batch processing of data and analytical processing for statistical use. The data is stored as key-value pairs where the value part contains columns of related data.
	
	\item \textbf{Graph databases} \\
	Graph databases are primarily used to describe relationships between data. Data is stored as nodes and edges. Nodes contain key-value pairs of data and edges describe the relationship between nodes.
\end{itemize}

%\subsubsection{Document databases}
%The document databases are designed to contain documents.
%The documents contains attribute name/value pairs.
%Attributes may vary between rows.
%To retrieve data it is possible to search both on the attribute and the value.
%
%Primary use include storing actual documents like emails and blog posts, or storage of semi-structured and aggregate data.
%
%\begin{figure}
%	\centering
%
%	\begin{tikzpicture}
%		\node[draw, rectangle, minimum height=4.5cm] (a) {
%			\begin{tabular}{c l}
%				\{ & \\
%				& ``ID'': 1, \\
%				& ``Firstname'': ``Thomas'', \\
%				& ``Lastname'': ``Steffensen'', \\
%				& ``Age'': 27, \\
%				& ``Zip'': ``8000'', \\
%				& ``City'': ``Aarhus'' \\
%				\} &
%			\end{tabular}};
%	\end{tikzpicture}
%
%	\captionsetup{format=plain,font=footnotesize,labelfont={bf,defaultCapFont},labelsep=quad,singlelinecheck=no}
%		\caption[Document store]{
%			\label{fig:DocumentStore}
%			\footnotesize{%
%				Document store structure.
%			} 
%	}
%\end{figure}
%
%% \begin{figure}
%% 	\centering
%% 	\includegraphics[scale=0.8]{Document.png} 
%% 	\captionsetup{format=plain,font=footnotesize,labelfont={bf,defaultCapFont},labelsep=quad,singlelinecheck=no}
%% 	\caption[Document store]{
%% 		\label{fig:DocumentStore}
%% 		\footnotesize{%
%% 			Document store structure.
%% 		} 
%% 	}
%% \end{figure}
%
%\subsubsection{Key-value stores}
%The key-value stores can be compared to a hashmap since every entry has a key and an associated value. 
%To retrieve data you search for the key. 
%The values can contain any kind of data from simple text to lists or documents.
%
%Primary use includes fast lookup for instance for user sessions or product lists.
%
%\begin{figure}
%	\centering
%
%	\begin{tikzpicture} [
%			diagram item/.style={
%				minimum width=3cm,
%				minimum height=1cm,
%				draw,
%				rectangle
%			}
%		]
%		\node[diagram item] (a) {Key: User1};
%		\node[diagram item, right=.5cm of a] (b) {Value: Stefan};
%
%		\node[diagram item, below=.2cm of a] (c) {Key: User2};
%		\node[diagram item, right=.5cm of c] (d) {Value: Thomas};
%
%	    \draw[arrows=->] (a) to (b);
%	    \draw[arrows=->] (c) to (d);
%	\end{tikzpicture}
%
%	\captionsetup{format=plain,font=footnotesize,labelfont={bf,defaultCapFont},labelsep=quad,singlelinecheck=no}
%		\caption[Key-value store]{
%			\label{fig:KeyValueStore}
%			\footnotesize{%
%				Key-value store structure.
%			} 
%		}
%\end{figure}
%
%% \begin{figure}
%% 	\centering
%% 	\includegraphics[scale=0.8]{KeyValue.png} 
%% 	\captionsetup{format=plain,font=footnotesize,labelfont={bf,defaultCapFont},labelsep=quad,singlelinecheck=no}
%% 	\caption[Key-value store]{
%% 		\label{fig:KeyValueStore}
%% 		\footnotesize{%
%% 			Key-value store structure.
%% 		} 
%% 	}
%% \end{figure}
%
%\subsubsection{Column-family stores}
%Column-family stores keep related data stored together.
%A column-family object consists of a key-value pair where the value contains columns of related data.
%
%Primary use includes distributed data storage, batch processing of data and analytical processing for statistical use.
%
%\begin{figure}
%	\centering
%
%	\begin{tikzpicture} [
%			diagram item/.style={
%				minimum width=3cm,
%				minimum height=1cm,
%				draw,
%				rectangle
%			},
%			diagram bigitem/.style={
%				minimum width=3cm,
%				minimum height=2cm,
%				draw,
%				rectangle
%			}
%		]
%
%
%		\node (a) {Row key};
%		\node [right=5.7cm of a] (b) {Columns};
%		\node[diagram bigitem, below=0cm of a] (c) {User1};
%		
%		%Attribute names
%		\node[diagram item, right=1.5cm of c, anchor=south] (d) {Firstname};
%		\node[diagram item, right=0cm of d] (e) {Lastname};
%		\node[diagram item, right=0cm of e] (f) {Age};
%		\node[diagram item, right=0cm of f] (g) {Zip};
%
%		%Attribute values
%		\node[diagram item, below=0cm of d] (h) {Thomas};
%		\node[diagram item, below=0cm of e] (i) {Steffensen};
%		\node[diagram item, below=0cm of f] (j) {27};
%		\node[diagram item, below=0cm of g] (k) {8000};
%
%		\node[diagram bigitem, below=.2cm of c] (l) {User2};
%		
%		%Attribute names
%		\node[diagram item, right=1.5cm of l, anchor=south] (m) {Firstname};
%		\node[diagram item, right=0cm of m] (n) {Zip};
%
%		%Attribute values
%		\node[diagram item, below=0cm of m] (o) {Thomas};
%		\node[diagram item, below=0cm of n] (p) {8000};
%	\end{tikzpicture}
%
%	\caption[Column-family store]{
%		\label{fig:ColumnFamilyStore}
%		\footnotesize{%
%			Column-family store structure.
%		} 
%	}
%\end{figure}
%
%% \begin{figure}
%% 	\centering
%% 	\includegraphics[scale=0.8]{ColumnFamily.png} 
%% 	\captionsetup{format=plain,font=footnotesize,labelfont={bf,defaultCapFont},labelsep=quad,singlelinecheck=no}
%% 	\caption[Column-family store]{
%% 		\label{fig:ColumnFamilyStore}
%% 		\footnotesize{%
%% 			Column-family store structure.
%% 		} 
%% 	}
%% \end{figure}
%
%\subsubsection{Graph databases}
%Graph databases divides data according to nodes and relations between nodes.
%Each node in the graph contains key-value pairs of data, and each edge describes a relationship to another node.
%Graph databases are optimized for traversal of relationships between nodes, not for data aggregation or analysis.
%
%Primary use includes pattern detection and mapping of networks.
%
%\begin{figure}
%	\centering
%	\begin{tikzpicture} [
%			diagram item/.style={
%				minimum width=5.3cm,
%				minimum height=1.5cm,
%				draw,
%				rectangle
%			}
%		]
%
%		\node[diagram item] (a) {
%			\begin{tabular}{rl}
%				Firstname:&Thomas \\
%				Lastname:&Steffensen
%			\end{tabular}};
%
%		\node[diagram item, right=3 of a] (b) {
%			\begin{tabular}{rl}
%				Firstname:&Stefan \\
%				Age:&27
%			\end{tabular}};
%
%		\node[diagram item, below=3 of a] (c) {
%			\begin{tabular}{rl}
%				Firstname:&Mette \\
%				Occupation:&Student
%			\end{tabular}};
%
%		\node[diagram item, right=3 of c] (d) {
%			\begin{tabular}{rl}
%				Name:&Aarhus Universitet
%			\end{tabular}};
%
%		\path (a) -- node[sloped] (knows) {knows} (b);
%		\path (a) -- node[sloped] (married) {married to} (c);
%		\path (a) -- node[sloped] (studies1) {studies at} (d);
%		\path (b) -- node[sloped] (studies2) {studies at} (d);
%
%	    \draw[->] (a)--(knows)--(b);
%	    \draw[->] (a)--(married)--(c);
%	    \draw[->] (a)--(studies1)--(d);
%	    \draw[->] (b)--(studies2)--(d);
%	\end{tikzpicture}
%\end{figure}
%
%% \begin{figure}
%% 	\centering
%% 	\includegraphics[scale=0.8]{Graph.png} 
%% 	\captionsetup{format=plain,font=footnotesize,labelfont={bf,defaultCapFont},labelsep=quad,singlelinecheck=no}
%% 	\caption[Graph store]{
%% 		\label{fig:GraphStore}
%% 		\footnotesize{%
%% 			Graph store structure.
%% 		} 
%% 	}
%% \end{figure}

\subsection{Relational storage that scale, NewSQL}
\label{sec:newsql}
NewSQL data stores aim to bring the relational data model into the world of horizontal scalability and flexible data models while maintaining the ACID properties and transactions of the traditional RDBMS~\cite{Cattell:ScalableSQLAndNoSQLDataStores}.
This is obtained by implementing a completely new architecture~\cite{CORBETT:SpannerGooglesGloballyDistributedDatabase}.
Starting from nothing with a new architecture allows the NewSQL data stores to be designed to take advantage of the distributed paradigms and to incorporate more flexibility into the schema structure.
The key differences between traditional SQL data stores and the NewSQL data stores are therefore found in the way the NewSQL data stores are built for scalability and throughput.
They try to avoid the major performance barriers which are locking, write-ahead logging, buffer pool overhead and latching~\cite{Stonebraker:NewSQLvsNoSQLForNewOLTP}.

Locking can be avoided by performing transactions in timestamp order or using multi version concurrency control.
Write-ahead logging can be avoided by doing automatic replication and failover.
To avoid buffer pool overhead the NewSQL data stores can run in main memory, either entirely or have a hot store in memory for active data and a cold store on disk for stale data.
To avoid latching transactions can be run single-threaded, meaning transactions must run to completion without descheduling.

Using some or all of these upgrades the NewSQL data stores can achieve higher throughput than the traditional SQL data stores.
Other features like distributed concurrency control and distributed query processing allows the NewSQL data stores to scale horizontally.

NewSQL data stores are divided into three categories~\cite{Prasanns:NewSQLTheNewWayToHandleBigData}:

\begin{itemize}
\item New databases: Completely new systems designed for scalability and throughput.
\item New MySQL storage engine: Keep the existing MySQL interface and redesign the storage engine in order to achieve scalability.
\item Transparent clustering and sharding: Provide extra features for transparent clustering and sharding on top of existing database systems. 
\end{itemize}

\subsection{SQL, NoSQL or NewSQL?}
This thesis aims to create a decentralized system with scalability and redundancy as the most important parameters.
This means that traditional SQL is not an option because of the poor scalability.

NoSQL and NewSQL has clear advantages because they are built as a consequence of the shortcomings of the traditional database systems when it comes to decentralized systems.
NoSQL has the advantage of high scalability and high throughput on data analysis, but the cost is a lack of ACID and transactions.
NewSQL promises to keep the ACID properties and transactions of the traditional database systems while simultaneously allowing horizontal scalability and high throughput.

Since both NoSQL and NewSQL seems like fitting technologies for data management in the Siemens case a further comparison between two state of the art implementations must be done in order to decide which technology will be the best suited.

\subsection{State of the art NoSQL}
To identify state of the NoSQL and NewSQL databases a website called  \inlineURL{db-engines.com}~\cite{db-engines} is used.
This website maintain a list of more than 200 different databases ranked by popularity. The list is updated monthly based on search engine popularity, discussion threads, job-offers, mentions on LinkedIn and tweets. This does not give the complete and objective ranking but it gives a pointer to the most popular database in their respective category.

Since we expect a stream of measured values from a wide range of parameters on every turbine a key-value store seems to be the obvious choice of NoSQL storage method. On top of the continuous stream of measured values aggregated measures must be obtained for the entire farm. This implies that custom software must be built to aggregate data from all the data stores and calculate aggregated values or that the data store has built in features for aggregation and calculation of aggregate values. Furthermore it is important that the database management system is able to do replication of data and automatic failover in order to achieve high availability.
Within the top 20 databases on the list we find 5 NoSQL databases:

\begin{itemize}
\item MongoDB~\cite{mongodb} ranked 5.
\item Cassandra~\cite{cassandra} ranked 9.
\item Redis~\cite{redis} ranked 12.
\item HBase~\cite{hbase} ranked 15.
\item Memcached~\cite{memcached} ranked 18.
\end{itemize}

Redis and Memcached are both key-value stores. Memcached is a very simply yet powerful distributed memory caching system.
It operates with a key for every entry and a value of raw data.
Memcached does not understand data structures so data must be serialized before upload.
Memcache does not support replication neither does it support advanced aggregate operations.

Redis is a more advanced key-value store.
It allows storing of data structures like lists, sets, hashmaps and so on.
Redis does not support aggregate operations.
Redis does not itself allow sharding but an extension called Redis Cluster do. This extension has just entered beta test phase and is not yet production ready.

Since both the key-value stores in the top 20 databases are used more like distributed memory than data stores they maintain a very simplistic approach to the interaction with data.
None of them support aggregate operations which is crucial for doing calculation over the entire farm.

The remaining NoSQL data stores are either document stores, MongoDB, or column-family stores, Cassandra and HBase.
Since data mostly has the structure of a parameter and a value it seems excessive to use a column-family data store. 
Column-family data stores are used for columns of related data which is sparse in this system.

That leaves the document store. MongoDB uses JSON-style documents to store data.
It can replicate and shard data. 
There is support for automatic failover if an instance is unavailable.
MongoDB supports data aggregation and mapreduce allowing aggregate operations before data is returned from the database.
In terms of availability, data distribution and aggregate operations MongoDB is the best of the NoSQL data stores.

\subsection{State of the art NewSQL}
Looking at the db-engines.com database list once more we find that the four highest ranking NewSQL databases are within the top 100 databases:

\begin{itemize}
\item SAP HANA~\cite{saphana} ranked 23.
\item Drizzle~\cite{drizzle} ranked 74.
\item NouDB~\cite{nuodb} ranked 83.
\item VoltDB~\cite{voltdb} ranked 90.
\end{itemize}

SAP HANA is developed by SAP.
It combines database and data processing in-memory.
SAP HANA supports planning, text processing and business analytics.
The platform has a lot of features but it is too excessive for this system.

Drizzle is an open source fork of MySQL reimplemented to support a plugin-based architecture.
The reimplementation is mainly focused on optimization for cloud infrastructure and web applications.
Lately it seems that the development has slowed and the project stalled.
The project homepage have several dead links and the last modification to the code was in may 2014.

NuoDB is a peer-to-peer oriented approach to the scalable database. Certain processes called Transaction Managers and Storage Managers share data on a peer-to-peer basis with no single point of failure.
This architecture supports sharding and replication.
What further separates NuoDB from the other NewSQL data stores is its ease of configuration and deployment. When a new instance is started it will automatically start communication with its peers. Administration of the database is done through a simple interface or administration can be set to run automatic.

VoltDB is a database built with the limitations of the traditional database systems in mind. Its focus is to avoid these limitations to achieve high throughput and scalability. The database is promoted on its high throughput compared to both traditional SQL databases and to other NoSQL and NewSQL databases.

Since all NewSQL databases are built for scalability, availability and use SQL as the query language they are able to scale, replicate and aggregate data.
In terms of ease of use NuoDB is the best choice but in terms of throughput VoltDB has some impressive benchmarks.
Since this system is a production system with feedback loops lasting only milliseconds throughput is important and that is why VoltDB is the best NewSQL alternative.

\subsection{Comparison of MongoDB and VoltDB}
Since MongoDB and VoltDB are the best fit to the Siemens Case in their respective categories a comparison of the two will determine which to use.
The comparison is based on the parameters presented in \cref{sec:databaseStorage}.
\begin{table}
	\begin{tabular}{l >{\centering}m{5cm} c}
		\hline
		\hline
		\textbf{Parameters} & \textbf{MongoDB} & \textbf{VoltDB} \\
		\hline
		\hline
		Scalability & \checkmark & \checkmark \\
		\hline
		Availability & \checkmark & \checkmark \\
		\hline
		Replication & \checkmark & \checkmark \\
		\hline
		Failover & \checkmark & \checkmark \\
		\hline
		Sharding & \checkmark & \checkmark \\
		\hline
		Aggregation & \checkmark & \checkmark \\
		\hline
		\hline
		\textbf{Additional parameters} & &\\
		\hline
		\hline
		Query language & JSON & SQL \\
		\hline
		Flexible schema & \checkmark & \text{x}  \\
		\hline
		Transactions & \text{x} & \checkmark  \\
		\hline
		ACID & \text{x} & \checkmark  \\
		\hline
		Industrial solutions & Orange, Forbes, Cisco, eBay, IBM, Microsoft, The Guardian & Schneider Electronics, Openet \\
		\hline
		\hline
	\end{tabular}
	
	\caption[MongoDB VoltDB]{
		\label{tab:mongovolt}
		\footnotesize{%
			Comparison of MongoDB and VoltDB.
		} 
	}
\end{table}

\subsection{Conclusion}
Choosing a data store these days is not easy. The data store business has been the disrupted by the amount of data generated by web 2.0. 
A number of new databases has spawned trying to solve the problems of traditional RDBMSs. 
The industry has not yet come to terms with the correct solution to the big data problem and therefore the best fitting solution as of now must be found.
MongoDB and VoltDB are two very different approaches to solve one problem.
MongoDB provides a tested solution that is very popular. VoltDB is the new solution promising even better features than MongoDB but the adaptation is still narrow.
For the Siemens case we choose MongoDB as the database.
The wider industry adaptation is a clear sign that MongoDB is a more mature and stable solution.
The popularity of MongoDB also means that a lot of resources and help is available.
MongoDB is a proven solution compared to the promising but yet untested VoltDB.
% !TeX spellcheck = en_GB
\chapter{Load balancing}

When dealing with redundant distributed systems, there exists more than one node capable of doing some work.
In such a system the workload needs to be distributed and balanced across all nodes.
This is done using a load balancer with a node balancing algorithm and some performance optimising features.

A node balancer is a service witch distributes incoming requests, among the services registered on the network.
The distribution is based on different policies like dividing packages or picking the one with most free CPU capacity.
The load balancer could be a single point of failure and there should therefore always be more than one. 

\begin{figure}[h!]
	\centering	
	\scalebox{0.7}{\begin{tikzpicture}[
	start chain=going right,
	diagram item/.style={
		minimum width=80pt,
		on chain,
		join
	},
	diagram item seperated/.style={
			minimum width=80pt,
			on chain
	}
]
\node [
	diagram item,
  label=center:Internet
] (Internet) {\includegraphics{Cisco_BW/cloud}};

%\node [
%	continue chain=going below,
%	diagram item,
%	label=right:Router
%] {\includegraphics{Cisco_BW/router}};

\node [
	start branch=1 going below right,
	diagram item seperated,
	label={[align=center]right:Load\\Balancer\\(Secondary)}
] (LB2) {\includegraphics{Cisco_BW/distributed_director}};

\node [
	continue chain=going below left,
	diagram item,
	label={[align=center]left:Load\\Balancer\\(Primary)}
] (LB1) {\includegraphics{Cisco_BW/distributed_director}};

\node [
	continue chain = going below right,
	diagram item,
	label={[align=center]right:Services in distrinbuted\\across the wind farm}
] (farm) {\includegraphics{Cisco_BW/web_cluster}};

\node [
	start branch=1 going below right,
	diagram item,
	label=below:Other interface
] {\includegraphics{Cisco_BW/PC}};

\node [
	start branch=1 going below left,
	diagram item,
	label=below:Http interface
] {\includegraphics{Cisco_BW/PC}};

\node [
	continue chain = going below,
	diagram item,
	label=below:Modbus interface
] {\includegraphics{Cisco_BW/PC}};

%Lines to/from LB2
\draw[loosely dotted] (LB1) -> (LB2) node[fill=white,midway]{heatbeat};
\draw[dashed] (Internet) -> (LB2);
\draw[dashed] (LB2) -> (farm);

\end{tikzpicture}}
	\captionsetup{format=plain,font=footnotesize,labelfont={bf,defaultCapFont},labelsep=quad,singlelinecheck=no}
	\caption[Distributed System with 2 load balancing nodes]{
		\label{fig:distributedCoputingSystem} 
		\footnotesize{%
			A Distributed System with 2 load balancing nodes.
		}
	}
\end{figure}

In this solution the load balancer needs to balance external connections to different protocols like HTTP and Modbus, however a solution witch can be extended to any restful protocol is needed. Also balancing of node roles depending on the amount incoming traffic on different interfaces will be needed.
Load balancers can also provide different features like bundling requests, security, discovering bad nodes and caching (Squid). This can offload the servers behind.

The following requirements to the system exists:
\begin{description}
	\item[Robustness]
	\item[Protocol flexible]
	\item[Distributed component] Must be a distributed component
\end{description}

Preferred features
\begin{itemize}
	\item support TCP Handoffs (for non restful applications)
\end{itemize}

\section{Levels of balancing}
\begin{description}
	\item[OSI 3] Network/IP %google says network layer LVS says transport layer
	\item[OSI 4] Network/IP
	\item[OSI 7] {Application level, like http balancing, allows balancing strategies based on url and user location.}
\end{description}

What we would like is a transport layer protocol.
\cite{Ludwig:SwarmIntelligenceGridLoadBalancing} Implements a particle swam based algorithm, and discuses quality parameters.

\section{Existing solutions}
\begin{description}
	\item[Linux Virtual Server: IPVS] Is implemented in the linux kernal version 2.4 and 2.6. Works at the IP level. Useed byt big sites sourceforge.net, layer 3.
	\item[Google Compute Engine: Load Balancer]: Proprietary. layer 3 and 7.
\end{description}
\chapter{Distributed computing}
As mentioned, low scalability is a problem in a Siemens windmill farm. The Wind Power Supervisor (WPS) and the HPPP does not scale well with the number of turbines, which introduces performance issues to the solution. Both in terms handling external requests, which is done by the WPS, but also when regulating the windmill farm through the HPPP. 

An example of this is the HPPP regulation sequence illustrated on \cref{fig:dataComputationSequence}. Today this sequence takes approximately 150 ms. Siemens wishes this time reduced to 10 ms. This is a major performance improvement and for that reason, performing the regulation sequence using a distributed database only is not enough, since reading/writing to the disk decreases performance.

When distributing the Wind Power Supervisor onto the turbines, the turbines obviously needs to be able to handle these external requests and windmill farm regulations. For the heavy tasks, in terms of CPU power, distributed computing becomes relevant as a way of improving performance by combining the CPU power residing inside the turbines to compute a common task.

In distributed computing, each node or process has its own local memory and communication happens via message passing~\cite{andrews2000foundations}. This chapter describes distributed computing communication paradigms, which are relevant to the Siemens case, and discusses which technology within the chosen paradigm that is the best for the Siemens case. 


\section{Message passing}

Message passing is a low-level communication paradigm, where processors communicate by sending messages via bidirectional channels. It's a highly used paradigm and other communication paradigms are usually implemented on top of an underlying message-passing system.  

With message passing being a low-level communication paradigm, the communication overhead is low compared to paradigms build on top of it. It is entirely up to the application developer to handle communication. This will in many cases result in better performance, which is the most compelling argument for choosing message passing as communication paradigm. The problem with it being up to the developer, is that the developer needs to deal with configurations setup, exception handling, who and when to communicate with, etc., when developing the application. This makes it hard to develop using message passing, compared to distributed shared memory, especially when dealing with more complex applications~\cite{lu1995message}. 


\section{Distributed shared memory}

Shared memory is an attractive paradigm for designing parallel and distributed systems. Applications can use shared memory as a tool for the entire system to share a common state. However for loose coupled distributed systems, no physically shared memory is available to support such a model. Distributed shared memory (DSM) is a way of providing physically distributed memory machines a shared memory abstraction, illustrated on \cref{fig:distributedSharedMemory}.

\begin{figure}
	\centering
	\includegraphics[width=0.8\textwidth,natwidth=610,natheight=642]{DistributedSharedMemory.jpg} 
	\captionsetup{format=plain,font=footnotesize,labelfont={bf,defaultCapFont},labelsep=quad,singlelinecheck=no}
	\caption[Distributed Computing System with 2 nodes]{
		\label{fig:distributedSharedMemory} 
		\footnotesize{%
			A distributed shared memory system with 2 nodes.
		}
	}
\end{figure}

The primary advantage of DSM over message passing is the shared memory abstraction provided. This gives the illusion of physically shared memory and allows developers to use the shared-memory paradigm, without having to think about communication mechanisms. However the abstraction also introduces overhead to the system, since DSM abstraction has limited knowledge of the application flow of the application, compared to communication via message passing. 

 

%DSM pass by reference

%In distribted system there might be scenarios in which a task waits for a service at the queue of one resource, while at the same time another resource which is capable of serving the task is idle. The purpose of a load balancing algorithm is to prevent these scenarios as much as possible.

%three phases.
%Information collection: Gathers info of workload
%decision making: Calc optimal data dist.
%data migration: Transfer excess amount of workload from on overloaded processor to another underloaded processor

%Centralized: Size of grid increases, keppeing all the inforation about the state of all the resources is a bottlebeck. Scalability becomes an issue. Page 281. 

%The benifits of this technique stems from Load Balancing
%State Broadcast Algorithm (SBA). Page 282

%Basic assumptions Page 289.

%Scalability and makespan (Y). Page 298, conclusion.


\section{Publish/subscribe}

Publish/subscribe is a messaging pattern where communication is interest based instead of address based. Messages are characterized into classes and sent by publishers, without knowledge of how many subscribers there may be. Nodes can then subscribe to one or more classes of interest, without knowledge of how many publishers there are, providing a more decoupled, scalable and flexible interaction model.

\begin{figure}
	\centering
	\includegraphics[width=0.9\textwidth,natwidth=610,natheight=642]{PublishSubscribe.jpg} 
	\captionsetup{format=plain,font=footnotesize,labelfont={bf,defaultCapFont},labelsep=quad,singlelinecheck=no}
	\caption[Distributed Computing System with 2 nodes]{
		\label{fig:publishSubscribe} 
		\footnotesize{%
			A simple publish/subscribe system.
		}
	}
\end{figure}

The publish/subscribe paradigm is event driven and corresponds to the observer design pattern, where subscribers are registered via keywords instead of registering their interest directly with the publishers. The paradigm relies on an event notification service providing storage and management for subscriptions and efficient delivery of events, as illustrated on \cref{fig:publishSubscribe}. The subscribers are notified subsequently of any event, generated by a publisher, matching the registered interest. The strength of this event-based communication is the full decoupling in time, space and synchronization between publishers and subscribers~\cite{eugster2003many}.

%Quality of service??

% DSM is only space and time decoupled but not sync, because consumers pull from shared space in a synchronous style


\section{Remote procedure call}

Remote procedure call (RPC) is a communications paradigm built for client/server architecture~\cite{Microsoft2003RPC}, which makes remote interactions appear the same way as local interactions. The goal is to make the process of executing code on a remote machine as simple as calling a local function~\cite{dusseau2014intro} by factoring out common tasks, such as security, synchronization, and data flow handling. This explains the paradigms popularity in distributed computing. However distribution cannot be made completely transparent to the application, because it gives rise to further types of potential failures, like communication failures, that have to be dealt with explicitly~\cite{coulouris2005distributed}. 

The idea of RPC is quite simple. When a remote procedure is invoked, the calling environment is suspended, the parameters are passed across the network to the environment where the procedure is to execute and the desired procedure is executed at that location. When execution is finished, return values are sent back to the calling environment, where execution resumes \cite{birrell1984implementing}.

A shortcoming of RPC is the strong coupling in time, space and synchronization. Although solutions have been presented to remove the synchronization coupling by future remote invocation. Remote method invocation is a paradigm where RPC as been applied to object-oriented contexts~\cite{eugster2003many}.

%Not appropriate for broadcasting

%Strong time coupled 
%sync coupled from the consumer side (waits for the return of the call, calling environment is suspended). Can be changed so sender does not expect reply (weak reliablity, no success or failure). Or return handle for sender to later request return value when needed (future remote invocation)

%Space coupling (remote reference to object)


%\section{Notification}
%
%The notification paradigm corresponds to the observer design pattern. It works by having subscribers register their interest directly with the publishers, which manages subscriptions and send events. It is usually implemented using two asynchronous invocations, in order to enforce synchronization decoupling: the first is sent by the client to the server, containing invocation arguments and a callback reference to the client, and the second is sent by the server to the client to return one or more replies. However publishers and subscribers remain coupled in time and space. Furthermore the communication management is left to the publisher. This can become a problem as the system grows in size \cite{eugster2003many}.

%Publish/Subscribe where subscribers register their interest directly with publishers, which manages subscriptions and send events.

%event driven

%\section{Message queuing}
%Message queuing is a message-centric approach that usually integrate some form of publish/subscribe transaction. It works by having producers append messages to a global FIFO or priority queue asynchronously and consumers dequeue them synchronously from that same queue, where messages can only be consumed by one consumer. At an interaction level message queues recall much of DSM, where producers feed messages to some global memory space. Similarly to DSM, producers and consumers are decoupled in both space and time, where synchronous decoupling is only present for the producers \cite{eugster2003many}.

%Global FIFO kø. Til hvis man er ligeglad med, hvem der tager opgaven??


%\begin{table}
%	\begin{tabular}{l >{\centering}m{5cm} c}
%		\hline
%		\hline
%		\textbf{Abstraction} & \textbf{Space} & \textbf{Time} & \textbf{Flow} \\
%		\hline
%		\hline
%		Message Passing & \checkmark & \checkmark \\
%		\hline
%		RPC/RMI & \checkmark & \checkmark \\
%		\hlines
%		Async. RPC/RMI & \checkmark & \checkmark \\
%		\hline
%		Future RPC/RMI & \checkmark & \checkmark \\
%		\hline
%		Notifications & \text{x}& \text{x} & \checkmark \\
%		\hline
%		DSM & \checkmark & \checkmark & P(\checkmark) \\
%		\hline
%		Message Queuing (PULL) & \checkmark & \checkmark & \text{P(} \checkmark \text{)} \\
%		\hline
%		Public/Subscribe & \checkmark & \checkmark & \checkmark \\
%		\hline
%		\hline
%	\end{tabular}
%	
%	\caption[MongoDB VoltDB]{
%		\label{tab:mongovolt}
%		\footnotesize{%
%			Comparison of MongoDB and VoltDB.
%		} 
%	}
%\end{table}

\section{Comparison with regards to the Siemens case}
Looking at the Siemens case (\cref{sec:SiemensCase}) the new distributed system must act as a single unit, be able to perform park regulations and scale easily with the number of turbines. Furthermore Siemens wish to remove single point of failures. With this in mind, the remote procedure call paradigm is not an option because it is tight coupled and build for a client/server architecture, which is exactly what Siemens is trying to avoid. One could imaging using a partial client/server architecture, with a communication hierarchy, however this would introduce some communication overhead~\cite{Yu1997JavaDSM} and single point of failures to the system.

Comparing DSM with message passing with regards to performance is not entirely fair since DSM is an abstraction built using message passing. However the comparison is necessary when considering what technology to use for a given system and in this case, the comparison results in a trade off between performance and the shared memory abstraction. 

Honghui~\cite{lu1995message} has studied the trade-off between message passing performance and the shared memory abstraction. He ported 12 different parallel program scenarios to a DSM system called TreadMarks and a message passing system called PVM and compared the two technologies with regards to programmability and performance. He argues that given DSM is an abstraction built on top of message passing, DSM cannot achieve better performance than message passing, given the larger software-overhead. Therefore the goal is to achieve the same performance as message passing using DSM. For 5 of the scenarios, TreadMarks performed within 10\% of PVM. For 6 of the programs the difference were between 10\% - 30\%. For the last scenario, PVM performed twice as well as TreadMarks. 

Honghui argues that the performance is dependent of the logical flow of the scenario. More messages and more data are sent in TreadMarks, explaining the performance differences. He gives the following reasons for the extra communication in TreadMarks:

\begin{itemize}
	\item Separation of synchronization and data transfer in TreadMarks. 
	\item Extra messages to request updates for data in the invalidate protocol used in TreadMarks.
	\item False sharing.
	\item Diff accumulation for migratory data in TreadMarks.
\end{itemize} 

%1) Seperation of synchronization
% Lazy release consistency: Against data races (which may result uin wrong results). Only the next processor that acquires the lock can access x --> only that processor is informormed of the change to x --> reduce message traffic. Ex: Barriers - No processor overites values before all processors have read the value computed in the previous interation.

%2) Extra messages to request updates for data in the invalidate protocol used in TreadMarks
% Memory page change communicatin. Modified pages are inviladated after an acquire. Later access causes access miss, which in turn causes installation of an up-to-date copy of the page.

%3) False sharing
% To objects er allokerede i samme memory page og de skrives til samtidig --> force update af page --> overhead

%4) diff accumulation for migratory data in TreadMarks
% Multiple-writer protocol to allow wrinting on same page at the same time. Uses a diff algorithm to reduce false sharing effects.

Honghui concludes that the performance of a well optimized DSM system is comparable to that of a message passing system. Furthermore, development of systems with complex communication patterns takes a lot less effort using the DSM paradigm.

In contrast to Honghui, Stumm and Zhou~\cite{stumm1990algorithms} argues that applications using DSM can in fact outperform their message passing counterparts, in a few cases. They argue, that this is possible for the following reasons:

\begin{itemize}
 	\item DSM algorithms typically move data on demand as they are being accessed, which spreads communication load over a longer period of time, allowing for a greater degree of concurrency. If for example a node uses the shared memory more than others, the node does not need to communicate for every write operation made to the shared memory.
 	\item For DSM algorithms that sends data in large blocks, communication overhead is reduced. 
\end{itemize} 

Looking at the Siemens case the two major factors for the communication paradigm choice are scalability and availability.
 With that in mind, BLABLA is not an option because of .. 

Message queuing and RMI offers feature which  



\section{Conclusion}

%Ens
%Valg med vægt på arkitektur og development tid. 
%RMI fravalg
\section{Database storage}
\label{sec:databaseStorage}
Data storage is important in a system like a wind farm.
A lot of data must be persisted like weather data, health data of the different parts of every turbine and production data.
These records are important both for immediate use to view the current state of the system but also for review in the future for instance to predict weather trends or replace worn down parts of turbine before they break completely.

Currently data is aggregated from each turbine and stored on a central node.
This node will over time aggregate hundreds of gigabytes of information.
The data on the node is secured by backup but it is still a single point of failure.
Take out the data storage node or the communication to it and a lot of information will be lost.

By distributing the data of the system between all the connected nodes we achieve better redundancy because the data is present on many different nodes.
Should a node become unavailable another node can communicate the same data in effect strengthening the availability of the system.

This chapter contains a description of a number of relevant data storage technologies and a discussion of which technology is the best suited for a system like the Siemens case presented in \cref{sec:SiemensCase}.

Storage technologies will be compared on a set of parameters that are relevant to the case presented in \cref{sec:SiemensCase}. The parameters are presented below in prioritized order:

\begin{enumerate}
\item \textbf{Scalability} \\
The storage technology must be able to scale horizontally in order to allow a varying number of nodes.

\item \textbf{Availability} \\
The data in the system must be available for processing at any time.

\item \textbf{Replication} \\
The data in the system must be replicated between nodes in order to avoid data loss should one node be damaged.
\item \textbf{Failover} \\
The data storage technology must be able to seamlessly switch from a damaged node to a working node if a failure occurs.

\item \textbf{Sharding} \\
The data storage technology must support automatic partitioning of datasets that are larger than the physical storage on one node.

\item \textbf{Aggregation} \\
The data storage technology must support aggregation of data.
\end{enumerate}

\subsection{Relational storage, SQL}
\label{sec:sql}
The traditional way of storing data is in a Relational Database Management System(RDBMS).
These databases rely on a schema to arrange data in tables and their relations.
Using SQL it is easy to query data and to do aggregate operations.
RDBMSs support ACID transactions which ensures operations in the database are processed reliably.

%A shortcoming of the RDBMSs is the problem with object-relational mapping also known as the Impedance Mismatch Problem\cite{Fowler:IntroNoSQL, Neward:TheVietnamOfComputerScience}.
%The relational structure of the RDBMSs does not map well to the object-oriented structure the most popular programming languages encourage.
%Often an object is an aggregate of a number of attributes.
%In the context of the object-oriented program the object is seen as one entity.
%In the context of the RDBMS the attributes of the object-oriented object is often scattered between multiple tables in the database to ensure consistency and avoid duplicate data.
%This mismatch between object representation and relational representation can cause both performance problems, the JOIN operation in SQL is very costly, as well as considerable development time spent mapping one structure to the other.
%The performance problem multiplies in a distributed database if the RDBMS must do JOIN operations across the network in order to aggregate data.

A problem with a traditional RDBMS is that they are designed for vertical scaling~\cite{Atzeni:TheRelationalModelIsDead}. If a traditional RDBMS has problems handling data the solution is to add a bigger harddrive or invest in a faster CPU. This makes sense in a world were hardware is very expensive like it was when the traditional RDBMSs saw the light of day~\cite{Stonebraker:TheEndOfAnArchitecturalEra}. Today horizontal scaling is preferred. If a system has a problem with the data load add another machine or add five others if that is what it takes. Since horizontal scalability is a very important feature of the data storage system the traditional RDBMS is not a viable option.

\subsection{Schema-less storage, NoSQL}
\label{sec:nosql}
Since 2009 the schema-less storage methods have become increasingly popular.
Relational databases could no longer keep up with the task of storing and querying big data.
A new breed of schema-less storage systems became popular because they could handle some of the problems big data caused for the relational storage systems.
This new breed of databases are designed for horizontal scalability and without strict schemas allowing a more flexible data model. 
They are capable of handling large scale amounts of data both for data storage but also for analysis or batch operations.
The downside however is the lack of the ACID properties which results in decreased consistency and the lack of transactions. %TODO: Dårligt formuleret omskriv!

The schema-less databases can be divided roughly into four categories~\cite{Fowler:IntroNoSQL, Moniruzzaman:NoSQLDatabaseNewEraOfDatabasesForBigDataAnalysis}:

\begin{itemize}
	\item \textbf{Document databases} \\
	The document databases are primarily used to store semi structured data in the form of documents. The data is stored in attribute name-value pairs, where the attributes may vary between rows.
	
	\item \textbf{Key-value databases} \\
	The key-value databases are primarily used for fast lookup of data based on a key. The data is stored in key-value pairs.
	
	\item \textbf{Column-family databases} \\
	The column-family stores are primarily used for distributed data storage, batch processing of data and analytical processing for statistical use. The data is stored as key-value pairs where the value part contains columns of related data.
	
	\item \textbf{Graph databases} \\
	Graph databases are primarily used to describe relationships between data. Data is stored as nodes and edges. Nodes contain key-value pairs of data and edges describe the relationship between nodes.
\end{itemize}

%\subsubsection{Document databases}
%The document databases are designed to contain documents.
%The documents contains attribute name/value pairs.
%Attributes may vary between rows.
%To retrieve data it is possible to search both on the attribute and the value.
%
%Primary use include storing actual documents like emails and blog posts, or storage of semi-structured and aggregate data.
%
%\begin{figure}
%	\centering
%
%	\begin{tikzpicture}
%		\node[draw, rectangle, minimum height=4.5cm] (a) {
%			\begin{tabular}{c l}
%				\{ & \\
%				& ``ID'': 1, \\
%				& ``Firstname'': ``Thomas'', \\
%				& ``Lastname'': ``Steffensen'', \\
%				& ``Age'': 27, \\
%				& ``Zip'': ``8000'', \\
%				& ``City'': ``Aarhus'' \\
%				\} &
%			\end{tabular}};
%	\end{tikzpicture}
%
%	\captionsetup{format=plain,font=footnotesize,labelfont={bf,defaultCapFont},labelsep=quad,singlelinecheck=no}
%		\caption[Document store]{
%			\label{fig:DocumentStore}
%			\footnotesize{%
%				Document store structure.
%			} 
%	}
%\end{figure}
%
%% \begin{figure}
%% 	\centering
%% 	\includegraphics[scale=0.8]{Document.png} 
%% 	\captionsetup{format=plain,font=footnotesize,labelfont={bf,defaultCapFont},labelsep=quad,singlelinecheck=no}
%% 	\caption[Document store]{
%% 		\label{fig:DocumentStore}
%% 		\footnotesize{%
%% 			Document store structure.
%% 		} 
%% 	}
%% \end{figure}
%
%\subsubsection{Key-value stores}
%The key-value stores can be compared to a hashmap since every entry has a key and an associated value. 
%To retrieve data you search for the key. 
%The values can contain any kind of data from simple text to lists or documents.
%
%Primary use includes fast lookup for instance for user sessions or product lists.
%
%\begin{figure}
%	\centering
%
%	\begin{tikzpicture} [
%			diagram item/.style={
%				minimum width=3cm,
%				minimum height=1cm,
%				draw,
%				rectangle
%			}
%		]
%		\node[diagram item] (a) {Key: User1};
%		\node[diagram item, right=.5cm of a] (b) {Value: Stefan};
%
%		\node[diagram item, below=.2cm of a] (c) {Key: User2};
%		\node[diagram item, right=.5cm of c] (d) {Value: Thomas};
%
%	    \draw[arrows=->] (a) to (b);
%	    \draw[arrows=->] (c) to (d);
%	\end{tikzpicture}
%
%	\captionsetup{format=plain,font=footnotesize,labelfont={bf,defaultCapFont},labelsep=quad,singlelinecheck=no}
%		\caption[Key-value store]{
%			\label{fig:KeyValueStore}
%			\footnotesize{%
%				Key-value store structure.
%			} 
%		}
%\end{figure}
%
%% \begin{figure}
%% 	\centering
%% 	\includegraphics[scale=0.8]{KeyValue.png} 
%% 	\captionsetup{format=plain,font=footnotesize,labelfont={bf,defaultCapFont},labelsep=quad,singlelinecheck=no}
%% 	\caption[Key-value store]{
%% 		\label{fig:KeyValueStore}
%% 		\footnotesize{%
%% 			Key-value store structure.
%% 		} 
%% 	}
%% \end{figure}
%
%\subsubsection{Column-family stores}
%Column-family stores keep related data stored together.
%A column-family object consists of a key-value pair where the value contains columns of related data.
%
%Primary use includes distributed data storage, batch processing of data and analytical processing for statistical use.
%
%\begin{figure}
%	\centering
%
%	\begin{tikzpicture} [
%			diagram item/.style={
%				minimum width=3cm,
%				minimum height=1cm,
%				draw,
%				rectangle
%			},
%			diagram bigitem/.style={
%				minimum width=3cm,
%				minimum height=2cm,
%				draw,
%				rectangle
%			}
%		]
%
%
%		\node (a) {Row key};
%		\node [right=5.7cm of a] (b) {Columns};
%		\node[diagram bigitem, below=0cm of a] (c) {User1};
%		
%		%Attribute names
%		\node[diagram item, right=1.5cm of c, anchor=south] (d) {Firstname};
%		\node[diagram item, right=0cm of d] (e) {Lastname};
%		\node[diagram item, right=0cm of e] (f) {Age};
%		\node[diagram item, right=0cm of f] (g) {Zip};
%
%		%Attribute values
%		\node[diagram item, below=0cm of d] (h) {Thomas};
%		\node[diagram item, below=0cm of e] (i) {Steffensen};
%		\node[diagram item, below=0cm of f] (j) {27};
%		\node[diagram item, below=0cm of g] (k) {8000};
%
%		\node[diagram bigitem, below=.2cm of c] (l) {User2};
%		
%		%Attribute names
%		\node[diagram item, right=1.5cm of l, anchor=south] (m) {Firstname};
%		\node[diagram item, right=0cm of m] (n) {Zip};
%
%		%Attribute values
%		\node[diagram item, below=0cm of m] (o) {Thomas};
%		\node[diagram item, below=0cm of n] (p) {8000};
%	\end{tikzpicture}
%
%	\caption[Column-family store]{
%		\label{fig:ColumnFamilyStore}
%		\footnotesize{%
%			Column-family store structure.
%		} 
%	}
%\end{figure}
%
%% \begin{figure}
%% 	\centering
%% 	\includegraphics[scale=0.8]{ColumnFamily.png} 
%% 	\captionsetup{format=plain,font=footnotesize,labelfont={bf,defaultCapFont},labelsep=quad,singlelinecheck=no}
%% 	\caption[Column-family store]{
%% 		\label{fig:ColumnFamilyStore}
%% 		\footnotesize{%
%% 			Column-family store structure.
%% 		} 
%% 	}
%% \end{figure}
%
%\subsubsection{Graph databases}
%Graph databases divides data according to nodes and relations between nodes.
%Each node in the graph contains key-value pairs of data, and each edge describes a relationship to another node.
%Graph databases are optimized for traversal of relationships between nodes, not for data aggregation or analysis.
%
%Primary use includes pattern detection and mapping of networks.
%
%\begin{figure}
%	\centering
%	\begin{tikzpicture} [
%			diagram item/.style={
%				minimum width=5.3cm,
%				minimum height=1.5cm,
%				draw,
%				rectangle
%			}
%		]
%
%		\node[diagram item] (a) {
%			\begin{tabular}{rl}
%				Firstname:&Thomas \\
%				Lastname:&Steffensen
%			\end{tabular}};
%
%		\node[diagram item, right=3 of a] (b) {
%			\begin{tabular}{rl}
%				Firstname:&Stefan \\
%				Age:&27
%			\end{tabular}};
%
%		\node[diagram item, below=3 of a] (c) {
%			\begin{tabular}{rl}
%				Firstname:&Mette \\
%				Occupation:&Student
%			\end{tabular}};
%
%		\node[diagram item, right=3 of c] (d) {
%			\begin{tabular}{rl}
%				Name:&Aarhus Universitet
%			\end{tabular}};
%
%		\path (a) -- node[sloped] (knows) {knows} (b);
%		\path (a) -- node[sloped] (married) {married to} (c);
%		\path (a) -- node[sloped] (studies1) {studies at} (d);
%		\path (b) -- node[sloped] (studies2) {studies at} (d);
%
%	    \draw[->] (a)--(knows)--(b);
%	    \draw[->] (a)--(married)--(c);
%	    \draw[->] (a)--(studies1)--(d);
%	    \draw[->] (b)--(studies2)--(d);
%	\end{tikzpicture}
%\end{figure}
%
%% \begin{figure}
%% 	\centering
%% 	\includegraphics[scale=0.8]{Graph.png} 
%% 	\captionsetup{format=plain,font=footnotesize,labelfont={bf,defaultCapFont},labelsep=quad,singlelinecheck=no}
%% 	\caption[Graph store]{
%% 		\label{fig:GraphStore}
%% 		\footnotesize{%
%% 			Graph store structure.
%% 		} 
%% 	}
%% \end{figure}

\subsection{Relational storage that scale, NewSQL}
\label{sec:newsql}
NewSQL data stores aim to bring the relational data model into the world of horizontal scalability and flexible data models while maintaining the ACID properties and transactions of the traditional RDBMS~\cite{Cattell:ScalableSQLAndNoSQLDataStores}.
This is obtained by implementing a completely new architecture~\cite{CORBETT:SpannerGooglesGloballyDistributedDatabase}.
Starting from nothing with a new architecture allows the NewSQL data stores to be designed to take advantage of the distributed paradigms and to incorporate more flexibility into the schema structure.
The key differences between traditional SQL data stores and the NewSQL data stores are therefore found in the way the NewSQL data stores are built for scalability and throughput.
They try to avoid the major performance barriers which are locking, write-ahead logging, buffer pool overhead and latching~\cite{Stonebraker:NewSQLvsNoSQLForNewOLTP}.

Locking can be avoided by performing transactions in timestamp order or using multi version concurrency control.
Write-ahead logging can be avoided by doing automatic replication and failover.
To avoid buffer pool overhead the NewSQL data stores can run in main memory, either entirely or have a hot store in memory for active data and a cold store on disk for stale data.
To avoid latching transactions can be run single-threaded, meaning transactions must run to completion without descheduling.

Using some or all of these upgrades the NewSQL data stores can achieve higher throughput than the traditional SQL data stores.
Other features like distributed concurrency control and distributed query processing allows the NewSQL data stores to scale horizontally.

NewSQL data stores are divided into three categories~\cite{Prasanns:NewSQLTheNewWayToHandleBigData}:

\begin{itemize}
\item New databases: Completely new systems designed for scalability and throughput.
\item New MySQL storage engine: Keep the existing MySQL interface and redesign the storage engine in order to achieve scalability.
\item Transparent clustering and sharding: Provide extra features for transparent clustering and sharding on top of existing database systems. 
\end{itemize}

\subsection{SQL, NoSQL or NewSQL?}
This thesis aims to create a decentralized system with scalability and redundancy as the most important parameters.
This means that traditional SQL is not an option because of the poor scalability.

NoSQL and NewSQL has clear advantages because they are built as a consequence of the shortcomings of the traditional database systems when it comes to decentralized systems.
NoSQL has the advantage of high scalability and high throughput on data analysis, but the cost is a lack of ACID and transactions.
NewSQL promises to keep the ACID properties and transactions of the traditional database systems while simultaneously allowing horizontal scalability and high throughput.

Since both NoSQL and NewSQL seems like fitting technologies for data management in the Siemens case a further comparison between two state of the art implementations must be done in order to decide which technology will be the best suited.

\subsection{State of the art NoSQL}
To identify state of the NoSQL and NewSQL databases a website called  \inlineURL{db-engines.com}~\cite{db-engines} is used.
This website maintain a list of more than 200 different databases ranked by popularity. The list is updated monthly based on search engine popularity, discussion threads, job-offers, mentions on LinkedIn and tweets. This does not give the complete and objective ranking but it gives a pointer to the most popular database in their respective category.

Since we expect a stream of measured values from a wide range of parameters on every turbine a key-value store seems to be the obvious choice of NoSQL storage method. On top of the continuous stream of measured values aggregated measures must be obtained for the entire farm. This implies that custom software must be built to aggregate data from all the data stores and calculate aggregated values or that the data store has built in features for aggregation and calculation of aggregate values. Furthermore it is important that the database management system is able to do replication of data and automatic failover in order to achieve high availability.
Within the top 20 databases on the list we find 5 NoSQL databases:

\begin{itemize}
\item MongoDB~\cite{mongodb} ranked 5.
\item Cassandra~\cite{cassandra} ranked 9.
\item Redis~\cite{redis} ranked 12.
\item HBase~\cite{hbase} ranked 15.
\item Memcached~\cite{memcached} ranked 18.
\end{itemize}

Redis and Memcached are both key-value stores. Memcached is a very simply yet powerful distributed memory caching system.
It operates with a key for every entry and a value of raw data.
Memcached does not understand data structures so data must be serialized before upload.
Memcache does not support replication neither does it support advanced aggregate operations.

Redis is a more advanced key-value store.
It allows storing of data structures like lists, sets, hashmaps and so on.
Redis does not support aggregate operations.
Redis does not itself allow sharding but an extension called Redis Cluster do. This extension has just entered beta test phase and is not yet production ready.

Since both the key-value stores in the top 20 databases are used more like distributed memory than data stores they maintain a very simplistic approach to the interaction with data.
None of them support aggregate operations which is crucial for doing calculation over the entire farm.

The remaining NoSQL data stores are either document stores, MongoDB, or column-family stores, Cassandra and HBase.
Since data mostly has the structure of a parameter and a value it seems excessive to use a column-family data store. 
Column-family data stores are used for columns of related data which is sparse in this system.

That leaves the document store. MongoDB uses JSON-style documents to store data.
It can replicate and shard data. 
There is support for automatic failover if an instance is unavailable.
MongoDB supports data aggregation and mapreduce allowing aggregate operations before data is returned from the database.
In terms of availability, data distribution and aggregate operations MongoDB is the best of the NoSQL data stores.

\subsection{State of the art NewSQL}
Looking at the db-engines.com database list once more we find that the four highest ranking NewSQL databases are within the top 100 databases:

\begin{itemize}
\item SAP HANA~\cite{saphana} ranked 23.
\item Drizzle~\cite{drizzle} ranked 74.
\item NouDB~\cite{nuodb} ranked 83.
\item VoltDB~\cite{voltdb} ranked 90.
\end{itemize}

SAP HANA is developed by SAP.
It combines database and data processing in-memory.
SAP HANA supports planning, text processing and business analytics.
The platform has a lot of features but it is too excessive for this system.

Drizzle is an open source fork of MySQL reimplemented to support a plugin-based architecture.
The reimplementation is mainly focused on optimization for cloud infrastructure and web applications.
Lately it seems that the development has slowed and the project stalled.
The project homepage have several dead links and the last modification to the code was in may 2014.

NuoDB is a peer-to-peer oriented approach to the scalable database. Certain processes called Transaction Managers and Storage Managers share data on a peer-to-peer basis with no single point of failure.
This architecture supports sharding and replication.
What further separates NuoDB from the other NewSQL data stores is its ease of configuration and deployment. When a new instance is started it will automatically start communication with its peers. Administration of the database is done through a simple interface or administration can be set to run automatic.

VoltDB is a database built with the limitations of the traditional database systems in mind. Its focus is to avoid these limitations to achieve high throughput and scalability. The database is promoted on its high throughput compared to both traditional SQL databases and to other NoSQL and NewSQL databases.

Since all NewSQL databases are built for scalability, availability and use SQL as the query language they are able to scale, replicate and aggregate data.
In terms of ease of use NuoDB is the best choice but in terms of throughput VoltDB has some impressive benchmarks.
Since this system is a production system with feedback loops lasting only milliseconds throughput is important and that is why VoltDB is the best NewSQL alternative.

\subsection{Comparison of MongoDB and VoltDB}
Since MongoDB and VoltDB are the best fit to the Siemens Case in their respective categories a comparison of the two will determine which to use.
The comparison is based on the parameters presented in \cref{sec:databaseStorage}.
\begin{table}
	\begin{tabular}{l >{\centering}m{5cm} c}
		\hline
		\hline
		\textbf{Parameters} & \textbf{MongoDB} & \textbf{VoltDB} \\
		\hline
		\hline
		Scalability & \checkmark & \checkmark \\
		\hline
		Availability & \checkmark & \checkmark \\
		\hline
		Replication & \checkmark & \checkmark \\
		\hline
		Failover & \checkmark & \checkmark \\
		\hline
		Sharding & \checkmark & \checkmark \\
		\hline
		Aggregation & \checkmark & \checkmark \\
		\hline
		\hline
		\textbf{Additional parameters} & &\\
		\hline
		\hline
		Query language & JSON & SQL \\
		\hline
		Flexible schema & \checkmark & \text{x}  \\
		\hline
		Transactions & \text{x} & \checkmark  \\
		\hline
		ACID & \text{x} & \checkmark  \\
		\hline
		Industrial solutions & Orange, Forbes, Cisco, eBay, IBM, Microsoft, The Guardian & Schneider Electronics, Openet \\
		\hline
		\hline
	\end{tabular}
	
	\caption[MongoDB VoltDB]{
		\label{tab:mongovolt}
		\footnotesize{%
			Comparison of MongoDB and VoltDB.
		} 
	}
\end{table}

\subsection{Conclusion}
Choosing a data store these days is not easy. The data store business has been the disrupted by the amount of data generated by web 2.0. 
A number of new databases has spawned trying to solve the problems of traditional RDBMSs. 
The industry has not yet come to terms with the correct solution to the big data problem and therefore the best fitting solution as of now must be found.
MongoDB and VoltDB are two very different approaches to solve one problem.
MongoDB provides a tested solution that is very popular. VoltDB is the new solution promising even better features than MongoDB but the adaptation is still narrow.
For the Siemens case we choose MongoDB as the database.
The wider industry adaptation is a clear sign that MongoDB is a more mature and stable solution.
The popularity of MongoDB also means that a lot of resources and help is available.
MongoDB is a proven solution compared to the promising but yet untested VoltDB.
% !TeX spellcheck = en_GB
\chapter{Load balancing}

When dealing with redundant distributed systems, there exists more than one node capable of doing some work.
In such a system the workload needs to be distributed and balanced across all nodes.
This is done using a load balancer with a node balancing algorithm and some performance optimising features.

A node balancer is a service witch distributes incoming requests, among the services registered on the network.
The distribution is based on different policies like dividing packages or picking the one with most free CPU capacity.
The load balancer could be a single point of failure and there should therefore always be more than one. 

\begin{figure}[h!]
	\centering	
	\scalebox{0.7}{\begin{tikzpicture}[
	start chain=going right,
	diagram item/.style={
		minimum width=80pt,
		on chain,
		join
	},
	diagram item seperated/.style={
			minimum width=80pt,
			on chain
	}
]
\node [
	diagram item,
  label=center:Internet
] (Internet) {\includegraphics{Cisco_BW/cloud}};

%\node [
%	continue chain=going below,
%	diagram item,
%	label=right:Router
%] {\includegraphics{Cisco_BW/router}};

\node [
	start branch=1 going below right,
	diagram item seperated,
	label={[align=center]right:Load\\Balancer\\(Secondary)}
] (LB2) {\includegraphics{Cisco_BW/distributed_director}};

\node [
	continue chain=going below left,
	diagram item,
	label={[align=center]left:Load\\Balancer\\(Primary)}
] (LB1) {\includegraphics{Cisco_BW/distributed_director}};

\node [
	continue chain = going below right,
	diagram item,
	label={[align=center]right:Services in distrinbuted\\across the wind farm}
] (farm) {\includegraphics{Cisco_BW/web_cluster}};

\node [
	start branch=1 going below right,
	diagram item,
	label=below:Other interface
] {\includegraphics{Cisco_BW/PC}};

\node [
	start branch=1 going below left,
	diagram item,
	label=below:Http interface
] {\includegraphics{Cisco_BW/PC}};

\node [
	continue chain = going below,
	diagram item,
	label=below:Modbus interface
] {\includegraphics{Cisco_BW/PC}};

%Lines to/from LB2
\draw[loosely dotted] (LB1) -> (LB2) node[fill=white,midway]{heatbeat};
\draw[dashed] (Internet) -> (LB2);
\draw[dashed] (LB2) -> (farm);

\end{tikzpicture}}
	\captionsetup{format=plain,font=footnotesize,labelfont={bf,defaultCapFont},labelsep=quad,singlelinecheck=no}
	\caption[Distributed System with 2 load balancing nodes]{
		\label{fig:distributedCoputingSystem} 
		\footnotesize{%
			A Distributed System with 2 load balancing nodes.
		}
	}
\end{figure}

In this solution the load balancer needs to balance external connections to different protocols like HTTP and Modbus, however a solution witch can be extended to any restful protocol is needed. Also balancing of node roles depending on the amount incoming traffic on different interfaces will be needed.
Load balancers can also provide different features like bundling requests, security, discovering bad nodes and caching (Squid). This can offload the servers behind.

The following requirements to the system exists:
\begin{description}
	\item[Robustness]
	\item[Protocol flexible]
	\item[Distributed component] Must be a distributed component
\end{description}

Preferred features
\begin{itemize}
	\item support TCP Handoffs (for non restful applications)
\end{itemize}

\section{Levels of balancing}
\begin{description}
	\item[OSI 3] Network/IP %google says network layer LVS says transport layer
	\item[OSI 4] Network/IP
	\item[OSI 7] {Application level, like http balancing, allows balancing strategies based on url and user location.}
\end{description}

What we would like is a transport layer protocol.
\cite{Ludwig:SwarmIntelligenceGridLoadBalancing} Implements a particle swam based algorithm, and discuses quality parameters.

\section{Existing solutions}
\begin{description}
	\item[Linux Virtual Server: IPVS] Is implemented in the linux kernal version 2.4 and 2.6. Works at the IP level. Useed byt big sites sourceforge.net, layer 3.
	\item[Google Compute Engine: Load Balancer]: Proprietary. layer 3 and 7.
\end{description}
\chapter{Distributed computing}
As mentioned, low scalability is a problem in a Siemens windmill farm. The Wind Power Supervisor (WPS) and the HPPP does not scale well with the number of turbines, which introduces performance issues to the solution. Both in terms handling external requests, which is done by the WPS, but also when regulating the windmill farm through the HPPP. 

An example of this is the HPPP regulation sequence illustrated on \cref{fig:dataComputationSequence}. Today this sequence takes approximately 150 ms. Siemens wishes this time reduced to 10 ms. This is a major performance improvement and for that reason, performing the regulation sequence using a distributed database only is not enough, since reading/writing to the disk decreases performance.

When distributing the Wind Power Supervisor onto the turbines, the turbines obviously needs to be able to handle these external requests and windmill farm regulations. For the heavy tasks, in terms of CPU power, distributed computing becomes relevant as a way of improving performance by combining the CPU power residing inside the turbines to compute a common task.

In distributed computing, each node or process has its own local memory and communication happens via message passing~\cite{andrews2000foundations}. This chapter describes distributed computing communication paradigms, which are relevant to the Siemens case, and discusses which technology within the chosen paradigm that is the best for the Siemens case. 


\section{Message passing}

Message passing is a low-level communication paradigm, where processors communicate by sending messages via bidirectional channels. It's a highly used paradigm and other communication paradigms are usually implemented on top of an underlying message-passing system.  

With message passing being a low-level communication paradigm, the communication overhead is low compared to paradigms build on top of it. It is entirely up to the application developer to handle communication. This will in many cases result in better performance, which is the most compelling argument for choosing message passing as communication paradigm. The problem with it being up to the developer, is that the developer needs to deal with configurations setup, exception handling, who and when to communicate with, etc., when developing the application. This makes it hard to develop using message passing, compared to distributed shared memory, especially when dealing with more complex applications~\cite{lu1995message}. 


\section{Distributed shared memory}

Shared memory is an attractive paradigm for designing parallel and distributed systems. Applications can use shared memory as a tool for the entire system to share a common state. However for loose coupled distributed systems, no physically shared memory is available to support such a model. Distributed shared memory (DSM) is a way of providing physically distributed memory machines a shared memory abstraction, illustrated on \cref{fig:distributedSharedMemory}.

\begin{figure}
	\centering
	\includegraphics[width=0.8\textwidth,natwidth=610,natheight=642]{DistributedSharedMemory.jpg} 
	\captionsetup{format=plain,font=footnotesize,labelfont={bf,defaultCapFont},labelsep=quad,singlelinecheck=no}
	\caption[Distributed Computing System with 2 nodes]{
		\label{fig:distributedSharedMemory} 
		\footnotesize{%
			A distributed shared memory system with 2 nodes.
		}
	}
\end{figure}

The primary advantage of DSM over message passing is the shared memory abstraction provided. This gives the illusion of physically shared memory and allows developers to use the shared-memory paradigm, without having to think about communication mechanisms. However the abstraction also introduces overhead to the system, since DSM abstraction has limited knowledge of the application flow of the application, compared to communication via message passing. 

 

%DSM pass by reference

%In distribted system there might be scenarios in which a task waits for a service at the queue of one resource, while at the same time another resource which is capable of serving the task is idle. The purpose of a load balancing algorithm is to prevent these scenarios as much as possible.

%three phases.
%Information collection: Gathers info of workload
%decision making: Calc optimal data dist.
%data migration: Transfer excess amount of workload from on overloaded processor to another underloaded processor

%Centralized: Size of grid increases, keppeing all the inforation about the state of all the resources is a bottlebeck. Scalability becomes an issue. Page 281. 

%The benifits of this technique stems from Load Balancing
%State Broadcast Algorithm (SBA). Page 282

%Basic assumptions Page 289.

%Scalability and makespan (Y). Page 298, conclusion.


\section{Publish/subscribe}

Publish/subscribe is a messaging pattern where communication is interest based instead of address based. Messages are characterized into classes and sent by publishers, without knowledge of how many subscribers there may be. Nodes can then subscribe to one or more classes of interest, without knowledge of how many publishers there are, providing a more decoupled, scalable and flexible interaction model.

\begin{figure}
	\centering
	\includegraphics[width=0.9\textwidth,natwidth=610,natheight=642]{PublishSubscribe.jpg} 
	\captionsetup{format=plain,font=footnotesize,labelfont={bf,defaultCapFont},labelsep=quad,singlelinecheck=no}
	\caption[Distributed Computing System with 2 nodes]{
		\label{fig:publishSubscribe} 
		\footnotesize{%
			A simple publish/subscribe system.
		}
	}
\end{figure}

The publish/subscribe paradigm is event driven and corresponds to the observer design pattern, where subscribers are registered via keywords instead of registering their interest directly with the publishers. The paradigm relies on an event notification service providing storage and management for subscriptions and efficient delivery of events, as illustrated on \cref{fig:publishSubscribe}. The subscribers are notified subsequently of any event, generated by a publisher, matching the registered interest. The strength of this event-based communication is the full decoupling in time, space and synchronization between publishers and subscribers~\cite{eugster2003many}.

%Quality of service??

% DSM is only space and time decoupled but not sync, because consumers pull from shared space in a synchronous style


\section{Remote procedure call}

Remote procedure call (RPC) is a communications paradigm built for client/server architecture~\cite{Microsoft2003RPC}, which makes remote interactions appear the same way as local interactions. The goal is to make the process of executing code on a remote machine as simple as calling a local function~\cite{dusseau2014intro} by factoring out common tasks, such as security, synchronization, and data flow handling. This explains the paradigms popularity in distributed computing. However distribution cannot be made completely transparent to the application, because it gives rise to further types of potential failures, like communication failures, that have to be dealt with explicitly~\cite{coulouris2005distributed}. 

The idea of RPC is quite simple. When a remote procedure is invoked, the calling environment is suspended, the parameters are passed across the network to the environment where the procedure is to execute and the desired procedure is executed at that location. When execution is finished, return values are sent back to the calling environment, where execution resumes \cite{birrell1984implementing}.

A shortcoming of RPC is the strong coupling in time, space and synchronization. Although solutions have been presented to remove the synchronization coupling by future remote invocation. Remote method invocation is a paradigm where RPC as been applied to object-oriented contexts~\cite{eugster2003many}.

%Not appropriate for broadcasting

%Strong time coupled 
%sync coupled from the consumer side (waits for the return of the call, calling environment is suspended). Can be changed so sender does not expect reply (weak reliablity, no success or failure). Or return handle for sender to later request return value when needed (future remote invocation)

%Space coupling (remote reference to object)


%\section{Notification}
%
%The notification paradigm corresponds to the observer design pattern. It works by having subscribers register their interest directly with the publishers, which manages subscriptions and send events. It is usually implemented using two asynchronous invocations, in order to enforce synchronization decoupling: the first is sent by the client to the server, containing invocation arguments and a callback reference to the client, and the second is sent by the server to the client to return one or more replies. However publishers and subscribers remain coupled in time and space. Furthermore the communication management is left to the publisher. This can become a problem as the system grows in size \cite{eugster2003many}.

%Publish/Subscribe where subscribers register their interest directly with publishers, which manages subscriptions and send events.

%event driven

%\section{Message queuing}
%Message queuing is a message-centric approach that usually integrate some form of publish/subscribe transaction. It works by having producers append messages to a global FIFO or priority queue asynchronously and consumers dequeue them synchronously from that same queue, where messages can only be consumed by one consumer. At an interaction level message queues recall much of DSM, where producers feed messages to some global memory space. Similarly to DSM, producers and consumers are decoupled in both space and time, where synchronous decoupling is only present for the producers \cite{eugster2003many}.

%Global FIFO kø. Til hvis man er ligeglad med, hvem der tager opgaven??


%\begin{table}
%	\begin{tabular}{l >{\centering}m{5cm} c}
%		\hline
%		\hline
%		\textbf{Abstraction} & \textbf{Space} & \textbf{Time} & \textbf{Flow} \\
%		\hline
%		\hline
%		Message Passing & \checkmark & \checkmark \\
%		\hline
%		RPC/RMI & \checkmark & \checkmark \\
%		\hlines
%		Async. RPC/RMI & \checkmark & \checkmark \\
%		\hline
%		Future RPC/RMI & \checkmark & \checkmark \\
%		\hline
%		Notifications & \text{x}& \text{x} & \checkmark \\
%		\hline
%		DSM & \checkmark & \checkmark & P(\checkmark) \\
%		\hline
%		Message Queuing (PULL) & \checkmark & \checkmark & \text{P(} \checkmark \text{)} \\
%		\hline
%		Public/Subscribe & \checkmark & \checkmark & \checkmark \\
%		\hline
%		\hline
%	\end{tabular}
%	
%	\caption[MongoDB VoltDB]{
%		\label{tab:mongovolt}
%		\footnotesize{%
%			Comparison of MongoDB and VoltDB.
%		} 
%	}
%\end{table}

\section{Comparison with regards to the Siemens case}
Looking at the Siemens case (\cref{sec:SiemensCase}) the new distributed system must act as a single unit, be able to perform park regulations and scale easily with the number of turbines. Furthermore Siemens wish to remove single point of failures. With this in mind, the remote procedure call paradigm is not an option because it is tight coupled and build for a client/server architecture, which is exactly what Siemens is trying to avoid. One could imaging using a partial client/server architecture, with a communication hierarchy, however this would introduce some communication overhead~\cite{Yu1997JavaDSM} and single point of failures to the system.

Comparing DSM with message passing with regards to performance is not entirely fair since DSM is an abstraction built using message passing. However the comparison is necessary when considering what technology to use for a given system and in this case, the comparison results in a trade off between performance and the shared memory abstraction. 

Honghui~\cite{lu1995message} has studied the trade-off between message passing performance and the shared memory abstraction. He ported 12 different parallel program scenarios to a DSM system called TreadMarks and a message passing system called PVM and compared the two technologies with regards to programmability and performance. He argues that given DSM is an abstraction built on top of message passing, DSM cannot achieve better performance than message passing, given the larger software-overhead. Therefore the goal is to achieve the same performance as message passing using DSM. For 5 of the scenarios, TreadMarks performed within 10\% of PVM. For 6 of the programs the difference were between 10\% - 30\%. For the last scenario, PVM performed twice as well as TreadMarks. 

Honghui argues that the performance is dependent of the logical flow of the scenario. More messages and more data are sent in TreadMarks, explaining the performance differences. He gives the following reasons for the extra communication in TreadMarks:

\begin{itemize}
	\item Separation of synchronization and data transfer in TreadMarks. 
	\item Extra messages to request updates for data in the invalidate protocol used in TreadMarks.
	\item False sharing.
	\item Diff accumulation for migratory data in TreadMarks.
\end{itemize} 

%1) Seperation of synchronization
% Lazy release consistency: Against data races (which may result uin wrong results). Only the next processor that acquires the lock can access x --> only that processor is informormed of the change to x --> reduce message traffic. Ex: Barriers - No processor overites values before all processors have read the value computed in the previous interation.

%2) Extra messages to request updates for data in the invalidate protocol used in TreadMarks
% Memory page change communicatin. Modified pages are inviladated after an acquire. Later access causes access miss, which in turn causes installation of an up-to-date copy of the page.

%3) False sharing
% To objects er allokerede i samme memory page og de skrives til samtidig --> force update af page --> overhead

%4) diff accumulation for migratory data in TreadMarks
% Multiple-writer protocol to allow wrinting on same page at the same time. Uses a diff algorithm to reduce false sharing effects.

Honghui concludes that the performance of a well optimized DSM system is comparable to that of a message passing system. Furthermore, development of systems with complex communication patterns takes a lot less effort using the DSM paradigm.

In contrast to Honghui, Stumm and Zhou~\cite{stumm1990algorithms} argues that applications using DSM can in fact outperform their message passing counterparts, in a few cases. They argue, that this is possible for the following reasons:

\begin{itemize}
 	\item DSM algorithms typically move data on demand as they are being accessed, which spreads communication load over a longer period of time, allowing for a greater degree of concurrency. If for example a node uses the shared memory more than others, the node does not need to communicate for every write operation made to the shared memory.
 	\item For DSM algorithms that sends data in large blocks, communication overhead is reduced. 
\end{itemize} 

Looking at the Siemens case the two major factors for the communication paradigm choice are scalability and availability.
 With that in mind, BLABLA is not an option because of .. 

Message queuing and RMI offers feature which  



\section{Conclusion}

%Ens
%Valg med vægt på arkitektur og development tid. 
%RMI fravalg
% !TeX spellcheck = en_US
\chapter{Prototype development}
The prototype in this project will consist of X virtual machines.



\section{Platform}

Windows
Pro
\begin{itemize}
	\item Well known to most people (easy to get going) 
	\item 
\end{itemize}

Bad
\begin{itemize}
	\item expensive
\end{itemize}



Linux
Pro
\begin{itemize}
	\item Market leader [?] Need Source
	\item Linux Virtual server
	\item Linux Containers
	\item Open source / free
	\item More configurable with respect to scheduler (non preemptive (default), preemptive voluntary, RT ) 
	\begin{itemize}
		\item \url{http://www.linuxtopia.org/online_books/linux_kernel/kernel_configuration/re152.html}
		\item \url{http://stackoverflow.com/questions/5174955/what-is-voluntary-preemption}
		\item \url{http://lwn.net/Articles/146861/}
		\item \url{https://www.osadl.org/uploads/media/ECE-2011-09.pdf}
		\item \url{http://www.linux.com/news/featured-blogs/200-libby-clark/710319-intro-to-real-time-linux-for-embedded-developers}
								
	\end{itemize}
\end{itemize}



%user: shared
%password: windfarm

%https://forums.virtualbox.org/viewtopic.php?f=6&t=63556&start=165
 




% !TeX spellcheck = en_US
\chapter{Discussion}

\section{Impact of turbine failure on the decentralized solution}
One of the main challenges of decentralized systems is to continue operation in the face of communication failure or node loss. This section address the \ref{PS:Q:Availability} problem of \cref{sec:problemStatement} which deal with the availability of the decentralized solution.

As presented in figures \ref{fig:exp:availability_kill1} to \ref{fig:exp:availability_kill15} the decentralized solution can handle the loss of turbines. A turbine is considered offline if it does not publish any new data to any other turbines within a 150 ms timespan as defined by the History QoS parameter described in \cref{sec:decen:ddsconf}.
This upper limit of 150 ms is chosen because this is the upper limit of the regulation cycle time in the current Simens system.
After the turbine has been detected as offline the remaining turbines in the wind farm will share the load of the missing turbines to keep the global power production of the park as close to the global setpoint as possible. The removal of one turbine from the windfarm is illustrated in \cref{fig:exp:availability_kill1}.

Removing several turbines from the system does not impede the regulation of the wind farm either. The remaining turbines will share the extra load between themselves and continue power production. The removal of several turbines are illustrated in figures \ref{fig:exp:availability_kill5} to \ref{fig:exp:availability_kill15}.

%The time it takes the decentralized solution to detect a turbine loss and recover power production is dependent on the following parameters:
%
%\begin{itemize}
%	\item Time of turbine loss detection, which is defined by the History QoS to 150 ms, $I$.
%	\item Time between regulation cycles, in the test cases presented in \cref{sec:res:availability} the time is set to 20 ms, $S$.
%	\item Time of calulation of a new setpoint, $C$.
%	\item Time for the turbine to regulate power production according to the new setpoint. In the turbines this process is simulated with a simple regulation that may take several regulation cycles to complete $R$.
%\end{itemize}
%
%Thus the time for recovery of the power production when a turbine failure occurs can be calculated as $T = I + S + C + R$ which translates to $150 ms + 20 ms + C + R = 170 + C + R ms$. This time can be reduced by lowering the time it takes to detect a turbine is offline which increases the likelihood of detecting a delayed turbine as offline, or by lowering the time between regulation cycles which increases the likelihood of cache reads.

Internal communication and regulation are only two parameters of availability though. To ensure that the wind farm is reachable from the outside world in the face of turbine failure, sets special requirements to the component chosen to handle external communication. A component capable of handling this specific problem has been described in \cref{cha:resourceManagement}.

We also need to address the problem of data loss when a turbine is unresponsive. A turbine has several control and measurement points which are continually logged to a local data store. Should a turbine break down the local data may be destroyed with the turbine. This again sets special requirements to the component handling data storage. In \cref{sec:databaseStorage} the requirements to the data storage component is described and MongoDB was presented as the best choice to solve the availability problem of data storage. MongoDB is capable of automatic replication of data between turbines such that data is still available should a turbine failure occur. As well as replication MongoDB provides automatic sharding of a database which enables global data to be stored across a number of nodes such that no single node contains all global data. Furthermore MongoDB supports aggregation of data across turbines to calculate aggregated values as for instance the global production of the wind farm.

Making the decentralized solution robust and able to handle the loss of turbines with regards to internal communication, regulation of power production, external communication and data storage increases the overall availability of the wind farm. 

\section{Number of turbines and the impact on regulation cycle time in the decentralized solution}
\label{sec:disc:turbinesVScycletime}
This section address the \ref{PS:Q:Performance} problem of \cref{sec:problemStatement}.

In the current Siemens system the regulation cycle time of a single Park Pilot scales linearly with the number of turbines.
The aim of the decentralized solution is to detach the regulation cycle time from the number of turbines. 
Looking at \cref{fig:exp:decen:turbines} we see that the decentralized solution seems to be independent of the number of turbines if the number of turbines is sufficiently low.
From 5 to 65 turbines the regulation cycle time is nearly constant at 20 ms.
The near constant regulation cycle time is caused by the fact that the regulation cycle in the decentralized solution is not forced to wait for data before running the regulation algorithm because data is continually shared between turbines. Adding a turbine to the decentralized solution adds an extra turbine state to factor into the setpoint calculation in the regulation algorithm. This and the added network traffic for the new turbine is the consequence of adding new turbines.

When looking at regulation cycle time of the decentralized solution we must also look at the number of cache reads.
As explained in \cref{sec:exp:performance} a cache read happens when a turbine does not provide a new turbine state package before the next regulation cycle is started.
This forces the regulation cycle to use old data read from cache.
Looking at \cref{fig:exp:decen:turbines_cache} we see that the average number of cache reads in the decentralized solution are below 2 and increasing slightly until we reach 65 turbines. From there the number of cache reads increase exponentially.

The increase in regulation cycle time and cache reads when the number of turbines reaches 65 can be explained by the fact that the network equipment of the test setup is approaching maximum throughput capacity which may cause lost or delayed network packages.
Since regulation cycle time in the decentralized system is dependent on the reception time of the oldest turbine state package as explained in \cref{sec:exp:performance} the loss or delay of network packages has a direct impact on regulation cycle time.
Similarly lost or delayed network packages increases the use of cached data. The increased regulation cycle time and cache reads are thus not a limitation of the decentralized solution but a limitation imposed by the test setup. In order to create a realistic comparison we must disregard the results that are a direct effect of the limitations of our test setup.

Disregarding the limitations imposed by the test setup we see that the regulation cycle time is constant. In terms of regulation cycle time the decentralized solution scales indefinitely. As stated above there is a small increase in regulation cycle time for every turbine added because the state of this turbine has to be taken into consideration when calculating new setpoints on all other turbines. This time addition is not visible on \cref{fig:exp:decen:turbines}. Thus the regulation cycle time will be affected by the addition of turbines but the effect is so small that it is indistinguishable by other factors. Looking at the raw test data 

Still disregarding the limitations imposed by the test setup when looking at the scale factor of the number of cache reads we see another result. The number of cache reads increases slowly with a factor of around 1 cache read for every 30 turbines added, which gives a scale factor of $1 / 30 = 0.033$.

The number of cache reads can be reduced by increasing the regulation cycle time as presented in \cref{fig:exp:decen:sleep-cache}.
Thus the factor deciding the time of the regulation cycle is the maximum number of average cache reads accepted for a single regulation cycle.

\section{Comparison of the decentralized solution and the centralized solution}
\label{sec:comp:decentralizedVScentralized}
The decentralized solution and the centralized solution is built with a very different architecture. This is reflected in the difference in the way the two solutions scale with the number of turbines. Looking at the figures for regulation cycle time in relation to the number of turbines for the decentralized and the centralized solution respectively we see that the figures reveals different trends. The test results from the centralized solution presented in \cref{fig:exp:cen:turbines} shows how adding turbines increases the regulation cycle time. From 5 to 30 turbines the regulation time of the centralized solution increase slowly. After 35 turbines the increase in regulation cycle time becomes steeper. For every 15 turbines added the regulation cycle time increases with approximately $25 ms$. The relation between regulation cycle time and the number of turbines is caused by the direct impact the number of turbines have on the regulation cycle. The Park Pilot of the centralized solution must use additional time on querying each added turbine as well as wait for their individual reply before performing the regulation algorithm. Thus the scalability of the centralized solution is linear with a factor of $25 / 15 = 1.667$.

The scalability of the decentralized solution, as described in \cref{sec:disc:turbinesVScycletime}, is small enough that it is indistinguishable from other factors in our test data and therefore we cannot calculate it. Thus the scalability of the decentralized solution is close to constant. This great improvement in scalability comes with a trade off in terms of cache reads. Adding additional turbines to the decentralized solution the number of cache reads increases with a factor of $0.033$ which is still an improvement compared to the scale factor of the centralized solution. Arguably comparing the scalability of the number of cache reads in the decentralized solution with the scalability of the regulation cycle in the centralized solution is not a viable way to compare the scalability of the two solutions, but comparing the scalability of the regulation cycle of both solutions will not be a fair comparison either since the improvements in scalability of the decentralized solution comes at the price of increased numbers of cache reads.

\section{Comparison of the decentralized solution and the current Siemens system}
This section address the \ref{PS:Q:Scalability} problem of \cref{sec:problemStatement}.
Comparing the decentralized solution and the current Siemens system directly is impossible given the differences in environment and architecture. Introducing a centralized solution is an attempt to bridge this gap. By comparing the decentralized solution to the centralized solution and measuring the improvements/demotions we can transfer these measures to the current Siemens system and an imagined decentralized implementation of the current Siemens system.

Looking at \cref{sec:comp:decentralizedVScentralized} there is a clear advantage in scalability when running a decentralized solution compared to a centralized. This is underlined by the improvement in regulation cycle time of the decentralized solution compared to the centralized solution.

Decentralizing the current Siemens system will enable improvements on other areas than regulation cycle time. By removing the centralized Park Pilots and the Wind Power Supervisor of the current Siemens solution the regulation of turbines, data storage and external communication of these must be decentralized and placed in the turbines themselves. This requires the use of software components that are able to handle regulation of turbines, data storage and external communication in a fashion such that if a turbine failure occurs other turbines can increase production to make up for the missing power production, provide access to the data collected on the failing turbine and handle external communication the failing turbine may have been handling. This increases the availability of the wind farm compared to the current Siemens system.

\begin{itemize}
	\item Test parameters: Our system vs Siemens system?
	\item Redundancy, up time, scalability...
	\item Solved problems (i.g. Single point of failure)
\end{itemize}
% !TeX spellcheck = en_US
\chapter{Conclusion}
Today, wind farms created by Siemens Wind Power has a centralized setup. Two central nodes exist in the wind farms, the Park Pilot and the Wind Power Supervisor. The Park Pilot is responsible for regulation of the power production of every turbine and for regulating the power production of the wind farm itself. The Wind Power Supervisor is responsible for data storage and external communication. The centralized approach to wind farm control poses a number of challenges for Siemens Wind Power.

The first challenge is a problem of scalability. The Park Pilot does not scale well with the number of turbines. This forces Siemens Wind Power to install several Park Pilots in one wind farm in order to properly regulate the power production of the turbines and the wind farm itself.

The second challenge is a problem of availability. Using centralized components as the Park Pilot and the Wind Power Supervisor introduces single points of failure in the wind farm. If a Park Pilot or the Wind Power Supervisor unexpectedly breaks down, parts of the wind farm are no longer available.

Thus, Siemens Wind Power wishes to increase the scalability and the availability of the wind farm setup by decentralizing the Park Pilots and the Wind Power Supervisor, such that their functionality is placed in the turbines.

In order to investigate the feasibility of decentralizing the Park Pilots and the Wind Power Supervisor the two components has been analyzed to identify the key services they provide.

The key service of the Park Pilots is to communicate with the turbines in order to regulate their accumulated power production according to the power production goal of the wind farm. Thus, to decentralize the Park Pilots the turbines must be able to communicate such that each turbine's power production can be regulated according to the power production goal of the wind farm without a centralized Park Pilot.

The key services of the Wind Power Supervisor is to facilitate external communication and to aggregate and calculate aggregate values for the wind farm. Thus, to decentralize the Wind Power Supervisor the turbines must be able to facilitate external communication as well as data aggregation across the wind farm.

Components able to provide decentralized communication, decentralized external communication and dezentralized data aggregation and storage in a decentralized environment has been identified. For decentralized communication between turbines the Data Distribution Service for Real Time Systems (DDS) is recommended. DDS supports communication using the publish/subscribe paradigm without a centralized broker. Furthermore, DDS is able to use multicast for transmission of network packages which limits the number of packages in the network thereby increasing the scalability of DDS. For decentralized external communication Linux Virtual Server Project (LVS) is recommended. LVS is built into the Linux kernel and is capable of redirecting traffic on the data-link layer. This makes for an outstanding performance in terms of connections handled per second. Similarly LVS facilitates direct communication between the turbine that is redirected to and the external connection initiator. For data storage and data aggregation MongoDB is recommended. MongoDB is able to scale horizontally as well as do replication of data such that in the event of a turbine failure data is not lost. MongoDB supports sharding of a database which is important as the data aggregated across the wind farm will accumulate to a size that no single turbine is able to store.

Using the identified components above we conclude that it is feasible to decentralize the current wind farm setup of a Siemens Wind Power wind farm.

Decentralizing the Park Pilots and the Wind Power Supervisor is expected to increase the availability of the wind farm. This puts certain requirements on the power regulation of the turbines, DDS, LVS and MongoDB.

Experiments show that the proposed decentralized solution is able to maintain wind farm power production in the face of turbine failures. Should a turbine become unavailable the remaining turbines are regulated such that they increase their power production in order to compensate for the unavailable turbine. Thus, internal communication handled by DDS is able to continue despite turbine failure.

LVS and MongoDB has not been implemented in the proposed decentralized solution, but they have been chosen partly because of their ability to uphold availability in a decentralized setup. LVS has the ability to detect load balancer failure by monitoring heartbeats from the load balancer that is currently handling external connections. Should the heartbeats cease to be sent another load balancer can take over and continue operation, by handling new external requests. MongoDB is capable of replication of databases across nodes. Thus, data collected on one turbine can be replicated to a number of other turbines. Should a turbine become unavailable the data collected on the unavailable turbine will still be accessible because it is replicated to other turbines.

By the experiment performed and the abilities of maintaining operation in spite of node failure of LVS and MongoDB, we conclude that removing one or more turbines from the system at runtime will not cause system failure. Decentralizing the current wind farm setup of a Siemens Wind Power wind farm will increase availability by removing the single points of failure embodied by the Park Pilots and the Wind Power Supervisor. By introducing failure resistant components capable of providing the same services as the Park Pilots and the Wind Power Supervisor the availability of the wind farm will be further increased.



Conclusion - discussion - perspektivering

Vis resultaternes indflydelse for Siemens!

%\chapter{Literature Review}\label{chapter1}


%\chapter{Your Title}

\section{You Can Underline This Headline - See Preamble }

\subsection{A Smaller Headline}

%Below you will find three different examples on figure configurations  
%\begin{figure}[t]
%\centering
%\includegraphics[trim=10 0 10
%10,clip,keepaspectratio,width=\textwidth]{Preprocessing/xt_and_yt2} 
%\vspace*{5mm}
%\includegraphics[trim=10 0 10
%10,clip,keepaspectratio,width=\textwidth]{Preprocessing/spectral4} 
%\caption[Spectral analysis of a signature]{The figure illustrates the $y$-coordinate and the $x$-coordinate as a function of time. The figures are contracted on the time axis, which means that time segments with pen ups have been removed. Below, a spectral analysis of the signals is illustrated.}
%\label{fig:xt_and_yt}
%\end{figure}

%\begin{SCfigure}[][t] 
%\includegraphics[trim=10 0 20
%10,clip,keepaspectratio,width=0.6\textwidth]{Preprocessing/Signature2} 
%\captionsetup{format=plain,font=footnotesize,labelfont={bf,red},labelsep=quad,singlelinecheck=no}
%\caption[A signature example from the SVC2004 database]{\label{fig:signature} \footnotesize{This figure illustrates one of the signatures of the SVC2004 database. The beginning of every stroke is marked with a red circle, and the order of stroke execution is indicated by the matching numbers.}} 
%\end{SCfigure}


%\begin{figure}[tb]
%\begin{narrow}{-\marginwidth}{-\marginwidth}
%\includegraphics[keepaspectratio,width=\linewidth]{Preprocessing/Pos_Covar2}
%\vspace*{3mm}
%\includegraphics[keepaspectratio,width=\linewidth]{Preprocessing/Neg_Covar2}
%\captionsetup{singlelinecheck=false,font=footnotesize,labelfont={bf,red},format=plain}
%\caption[Correlation of 2D Gaussians]{The upper figures illustrates positive correlation between two normal distributions, while the lower figures illustrate negative correlation. The covariance is a measure of the linear relationship between the two components}
%\label{fig:EX_Covar}
%\end{narrow}
%\end{figure}





%% !TeX spellcheck = en_US
\chapter{Conclusion}
Today, wind farms created by Siemens Wind Power has a centralized setup. Two central nodes exist in the wind farms, the Park Pilot and the Wind Power Supervisor. The Park Pilot is responsible for regulation of the power production of every turbine and for regulating the power production of the wind farm itself. The Wind Power Supervisor is responsible for data storage and external communication. The centralized approach to wind farm control poses a number of challenges for Siemens Wind Power.

The first challenge is a problem of scalability. The Park Pilot does not scale well with the number of turbines. This forces Siemens Wind Power to install several Park Pilots in one wind farm in order to properly regulate the power production of the turbines and the wind farm itself.

The second challenge is a problem of availability. Using centralized components as the Park Pilot and the Wind Power Supervisor introduces single points of failure in the wind farm. If a Park Pilot or the Wind Power Supervisor unexpectedly breaks down, parts of the wind farm are no longer available.

Thus, Siemens Wind Power wishes to increase the scalability and the availability of the wind farm setup by decentralizing the Park Pilots and the Wind Power Supervisor, such that their functionality is placed in the turbines.

In order to investigate the feasibility of decentralizing the Park Pilots and the Wind Power Supervisor the two components has been analyzed to identify the key services they provide.

The key service of the Park Pilots is to communicate with the turbines in order to regulate their accumulated power production according to the power production goal of the wind farm. Thus, to decentralize the Park Pilots the turbines must be able to communicate such that each turbine's power production can be regulated according to the power production goal of the wind farm without a centralized Park Pilot.

The key services of the Wind Power Supervisor is to facilitate external communication and to aggregate and calculate aggregate values for the wind farm. Thus, to decentralize the Wind Power Supervisor the turbines must be able to facilitate external communication as well as data aggregation across the wind farm.

Components able to provide decentralized communication, decentralized external communication and dezentralized data aggregation and storage in a decentralized environment has been identified. For decentralized communication between turbines the Data Distribution Service for Real Time Systems (DDS) is recommended. DDS supports communication using the publish/subscribe paradigm without a centralized broker. Furthermore, DDS is able to use multicast for transmission of network packages which limits the number of packages in the network thereby increasing the scalability of DDS. For decentralized external communication Linux Virtual Server Project (LVS) is recommended. LVS is built into the Linux kernel and is capable of redirecting traffic on the data-link layer. This makes for an outstanding performance in terms of connections handled per second. Similarly LVS facilitates direct communication between the turbine that is redirected to and the external connection initiator. For data storage and data aggregation MongoDB is recommended. MongoDB is able to scale horizontally as well as do replication of data such that in the event of a turbine failure data is not lost. MongoDB supports sharding of a database which is important as the data aggregated across the wind farm will accumulate to a size that no single turbine is able to store.

Using the identified components above we conclude that it is feasible to decentralize the current wind farm setup of a Siemens Wind Power wind farm.

Decentralizing the Park Pilots and the Wind Power Supervisor is expected to increase the availability of the wind farm. This puts certain requirements on the power regulation of the turbines, DDS, LVS and MongoDB.

Experiments show that the proposed decentralized solution is able to maintain wind farm power production in the face of turbine failures. Should a turbine become unavailable the remaining turbines are regulated such that they increase their power production in order to compensate for the unavailable turbine. Thus, internal communication handled by DDS is able to continue despite turbine failure.

LVS and MongoDB has not been implemented in the proposed decentralized solution, but they have been chosen partly because of their ability to uphold availability in a decentralized setup. LVS has the ability to detect load balancer failure by monitoring heartbeats from the load balancer that is currently handling external connections. Should the heartbeats cease to be sent another load balancer can take over and continue operation, by handling new external requests. MongoDB is capable of replication of databases across nodes. Thus, data collected on one turbine can be replicated to a number of other turbines. Should a turbine become unavailable the data collected on the unavailable turbine will still be accessible because it is replicated to other turbines.

By the experiment performed and the abilities of maintaining operation in spite of node failure of LVS and MongoDB, we conclude that removing one or more turbines from the system at runtime will not cause system failure. Decentralizing the current wind farm setup of a Siemens Wind Power wind farm will increase availability by removing the single points of failure embodied by the Park Pilots and the Wind Power Supervisor. By introducing failure resistant components capable of providing the same services as the Park Pilots and the Wind Power Supervisor the availability of the wind farm will be further increased.



Conclusion - discussion - perspektivering

Vis resultaternes indflydelse for Siemens!

\appendix
\addtocontents{toc}{\protect\setcounter{tocdepth}{0}}
\appendixpage
\appendix
\addtocontents{toc}{\protect\setcounter{tocdepth}{0}}
\chapter{Deriving Max Likelihood for GMM/HMM Using the EM Algorithm}\label{appendix1}

%\appendix
\chapter{Implementation of the Baum-Welch Algorithm}\label{appendix2}

%\appendix
%\addtocontents{toc}{\protect\setcounter{tocdepth}{0}}
\chapter{Feature Extraction for the HMM}\label{appendix3}


\backmatter
\renewcommand{\bibsection}{%
\chapter{\bibname}
\prebibhook}
\bibliographystyle{plain} % eller en anden stil
%\bibliographystyle{IEEEbib}
\bibliography{my-bibliography-file}
%\bibliography{minbib}
% print indeks hvis det er noget man anvender

\clearpage
\listoffigures
\clearpage
\listoftables
\chapter{Nomenclature}


\begin{table}[ht]
\captionsetup{singlelinecheck=false,labelsep=newline,justification=centering,font=footnotesize,labelfont={bf,defaultCapFont},format=plain}
\caption[Nomenclature]{\textsc{\ann{The Terminology used in this Thesis}}} % title of Table
\centering % used for centering table
\begin{tabular}{l l l} % centered columns (4 columns)
\hline\hline\\ %inserts double horizontal lines
 Acronym & Description \\ [0.5ex] % inserts table
%heading
\hline\\ % inserts single horizontal line

	SPOF & Single point of failure \\
	WPS &  Wind Power Supervisor\\[1ex] % [1ex] adds vertical space

\hline %inserts single line
\end{tabular}
\normalsize
\label{table:nomenclature} % is used to refer this table in the text
\end{table}




\printindex
\end{document}



