\chapter{Distributed computing}
As mentioned, low scalability is a problem in a Siemens windmill farm. The Wind Power Supervisor (WPS) and the Park Regulator does not scale well with the number of mills, which introduces performance issues to the solution. Both in terms handling external requests, which is done by the WPS, but also when regulating the park through the Park Regulator. 

An example of this is the Regulator calculation sequence illustrated on \cref{fig:dataComputationSequence}. Today this sequence takes approximately 150 ms. Siemens wishes this time reduced to 10 ms.

When distributing the Wind Power Supervisor onto the mills, the mills obviously needs to be able to handle these external requests and park regulations. For the heavy tasks, in terms of CPU power, distributed computing becomes relevant as a way of improving performance by combining the CPU power residing inside the mills to compute a common task. 

This chapter describes relevant distributed computing communication technologies and discusses which technology that is the best for the Siemens case. 


%In parallel computing, all processors may have access to a shared memory to exchange information between processors.[17]
%In distributed computing, each processor has its own private memory (distributed memory). Information is exchanged by passing messages between the processors.[18]


\section{Krav til distribueret computing. Hvad kræver det?}


\section{Holistic}