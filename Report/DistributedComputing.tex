\section{Distributed computing}

As mentioned, The Wind Power Supervisor (WPS) and the Park Pilot does not scale well with the number of turbines, which forces Siemens to set the regulation cycle time after worst case scenarios (see \cref{sec:SiemensCase} for details). Today this cycle time is set to 150 ms and Siemens wants the time reduced to 10 ms. This is a major performance improvement and as such not a strict demand from the Siemens case, meaning any reduction in the 150 ms cycle time will be accepted. The goal for the decentralized solution to the Siemens case is therefore to reduce the regulation cycle time as much as possible.

Looking at the park regulation algorithm \cref{sec:SiemensCase}, the reason for it being slow is primarily the communication overhead involved, when requesting set points from every turbine and sending new set points to every turbine, and that this communication overhead increases with the number of turbines involved with the regulation. So to reduce the regulation cycle time, this communication overhead needs to be reduced, by letting each turbine perform park regulations and calculate their own set points. 

%Removing the communication overhead from the regulation algorithm involves detaching sharing of data from the algorithm. In order to do this

For the Park Pilot to be able to compute a park regulation, the Park Pilot needs information about every turbine in the farm. In the same way, when decentralizing the Park Pilot, if a turbine is to compute park regulations, the turbine needs information about every other turbine. This means the windmill farm needs a global shared state, where information about every turbine in the farm is available to every turbine performing park regulations. 

%This is a major performance improvement and for that reason, performing the regulation sequence using a distributed database only is not enough, since reading/writing to the disk takes valuable milliseconds. Therefore regulation information needs to be kept in memory in order to keep regulation cycle time as low as possible. 

This chapter describes and discusses different distributed computing paradigms as a way of making a global state for the turbines to compute park regulations themselves. The goal is to find the paradigm best suited for the Siemens case. Furthermore, the chapter describes relevant technologies within the chosen paradigm and discusses which technology that is the best for the Siemens case.

The chapter uses the following definition by Andrews~\cite{andrews2000foundations} for distributed computing: \textit{In distributed computing, each node or process has its own local memory and communication happens via message passing.}



%Furthermore, when decentralizing the Wind Power Supervisor onto the turbines, the turbines obviously needs to be able to handle external data aggregation . For the heavy tasks, in terms of CPU power, distributed computing becomes relevant as a way of improving performance by combining the CPU power residing inside the turbines to compute a common task.

%
%In distributed computing, each node or process has its own local memory and communication happens via message passing~\cite{andrews2000foundations}. This means distributed computing is a way of having a global system state and keep relevant information, with regards to the global state, in memory, and thereby avoid read/write operations to the disk. 



\subsection{Message passing}

Message passing is a low-level communication paradigm, where processors communicate by sending messages via bidirectional channels. It is a highly used paradigm and other communication paradigms are usually implemented on top of an underlying message-passing system.  

With message passing being a low-level communication paradigm, the communication overhead is low compared to paradigms build on top of it. It is entirely up to the application developer to handle communication. This will in many cases result in better execution time, which is the most compelling argument for choosing message passing as communication paradigm. The problem with it being up to the developer, is that the developer needs to deal with configurations setup, such as sockets and marshaling, exception handling and who and when to communicate with when developing the application. This makes it hard to develop using message passing, especially when dealing with more complex applications~\cite{lu1995message}. 


\subsection{Distributed shared memory}

Shared memory is an attractive paradigm for designing parallel and distributed systems. Applications can use shared memory as a tool for the entire system to share a common state. However for loose coupled distributed systems, no physically shared memory is available to support such a model. Distributed shared memory (DSM) is a way of providing physically distributed memory machines a shared memory abstraction, illustrated on \cref{fig:distributedSharedMemory}.

\begin{figure}
	\centering
	\includegraphics[width=0.8\textwidth,natwidth=610,natheight=642]{DistributedSharedMemory.jpg} 
	\captionsetup{format=plain,font=footnotesize,labelfont={bf,defaultCapFont},labelsep=quad,singlelinecheck=no}
	\caption[Distributed Computing System with 2 nodes]{
		\label{fig:distributedSharedMemory} 
		\footnotesize{%
			A distributed shared memory system with 2 nodes.
		}
	}
\end{figure}

The primary advantage of DSM is the shared memory abstraction provided. This gives the illusion of physically shared memory and allows developers to use the shared-memory paradigm, without having to think about communication mechanisms. For this reason the DSM paradigm is fully decoupled in space, since producers and consumers of data remain anonymous to each other, and in time, since the producers needs no knowledge of future use of the data. Synchronization decoupling is achieved by some implementations of DSM, where each node keeps local copies of the shared data~\cite{guedes1993distributed}.

A downside to the DSM abstraction is that it introduces overhead to the system, since the DSM abstraction has limited knowledge of the application flow, compared to communication via message passing~\cite{lu1995message}. 

 

%DSM pass by reference

%In distribted system there might be scenarios in which a task waits for a service at the queue of one resource, while at the same time another resource which is capable of serving the task is idle. The purpose of a load balancing algorithm is to prevent these scenarios as much as possible.

%three phases.
%Information collection: Gathers info of workload
%decision making: Calc optimal data dist.
%data migration: Transfer excess amount of workload from on overloaded processor to another underloaded processor

%Centralized: Size of grid increases, keppeing all the inforation about the state of all the resources is a bottlebeck. Scalability becomes an issue. Page 281. 

%The benifits of this technique stems from Load Balancing
%State Broadcast Algorithm (SBA). Page 282

%Basic assumptions Page 289.

%Scalability and makespan (Y). Page 298, conclusion.


\subsection{Publish/subscribe}

Publish/subscribe is a messaging pattern where communication is interest based instead of address based. Messages are characterized into classes and sent by publishers, without knowledge of how many subscribers there may be. Nodes can then subscribe to one or more classes of interest, without knowledge of how many publishers there are, providing a more decoupled, scalable and flexible interaction model.

\begin{figure}
	\centering
	\includegraphics[width=0.9\textwidth,natwidth=610,natheight=642]{PublishSubscribe.jpg} 
	\captionsetup{format=plain,font=footnotesize,labelfont={bf,defaultCapFont},labelsep=quad,singlelinecheck=no}
	\caption[Distributed Computing System with 2 nodes]{
		\label{fig:publishSubscribe} 
		\footnotesize{%
			A simple publish/subscribe system.
		}
	}
\end{figure}

The publish/subscribe paradigm is event driven and corresponds to the observer design pattern, where subscribers are registered via keywords instead of registering their interest directly with the publishers. The paradigm relies on an event notification service providing storage and management for subscriptions and efficient delivery of events, as illustrated on \cref{fig:publishSubscribe}. The subscribers are notified subsequently of any event, generated by a publisher, matching the registered interest. The strength of this event-based communication is the full decoupling in time, space and synchronization between publishers and subscribers~\cite{eugster2003many}.

%Quality of service??

% DSM is only space and time decoupled but not sync, because consumers pull from shared space in a synchronous style


\subsection{Remote procedure call}

Remote procedure call (RPC) is a communications paradigm built for client/server architecture~\cite{Microsoft2003RPC}, which makes remote interactions appear the same way as local interactions. The goal is to make the process of executing code on a remote machine as simple as calling a local function~\cite{dusseau2014intro} by factoring out common tasks, such as security, synchronization, and data flow handling. This explains the paradigms popularity in distributed computing. However distribution cannot be made completely transparent to the application, because it gives rise to further types of potential failures, like communication failures, that have to be dealt with explicitly~\cite{coulouris2005distributed}. 

The idea of RPC is quite simple. When a remote procedure is invoked, the calling environment is suspended, the parameters are passed across the network to the environment where the procedure is to execute and the desired procedure is executed at that location. When execution is finished, return values are sent back to the calling environment, where execution resumes \cite{birrell1984implementing}.

A shortcoming of RPC is the strong coupling in time, space and synchronization. Although solutions have been presented to remove the synchronization coupling by future remote invocation. Remote method invocation is a paradigm where RPC as been applied to object-oriented contexts~\cite{eugster2003many}.

%Not appropriate for broadcasting

%Strong time coupled 
%sync coupled from the consumer side (waits for the return of the call, calling environment is suspended). Can be changed so sender does not expect reply (weak reliablity, no success or failure). Or return handle for sender to later request return value when needed (future remote invocation)

%Space coupling (remote reference to object)


%\section{Notification}
%
%The notification paradigm corresponds to the observer design pattern. It works by having subscribers register their interest directly with the publishers, which manages subscriptions and send events. It is usually implemented using two asynchronous invocations, in order to enforce synchronization decoupling: the first is sent by the client to the server, containing invocation arguments and a callback reference to the client, and the second is sent by the server to the client to return one or more replies. However publishers and subscribers remain coupled in time and space. Furthermore the communication management is left to the publisher. This can become a problem as the system grows in size \cite{eugster2003many}.

%Publish/Subscribe where subscribers register their interest directly with publishers, which manages subscriptions and send events.

%event driven

%\section{Message queuing}
%Message queuing is a message-centric approach that usually integrate some form of publish/subscribe transaction. It works by having producers append messages to a global FIFO or priority queue asynchronously and consumers dequeue them synchronously from that same queue, where messages can only be consumed by one consumer. At an interaction level message queues recall much of DSM, where producers feed messages to some global memory space. Similarly to DSM, producers and consumers are decoupled in both space and time, where synchronous decoupling is only present for the producers \cite{eugster2003many}.

%Global FIFO kø. Til hvis man er ligeglad med, hvem der tager opgaven??


%\begin{table}
%	\begin{tabular}{l >{\centering}m{5cm} c}
%		\hline
%		\hline
%		\textbf{Abstraction} & \textbf{Space} & \textbf{Time} & \textbf{Flow} \\
%		\hline
%		\hline
%		Message Passing & \checkmark & \checkmark \\
%		\hline
%		RPC/RMI & \checkmark & \checkmark \\
%		\hlines
%		Async. RPC/RMI & \checkmark & \checkmark \\
%		\hline
%		Future RPC/RMI & \checkmark & \checkmark \\
%		\hline
%		Notifications & \text{x}& \text{x} & \checkmark \\
%		\hline
%		DSM & \checkmark & \checkmark & P(\checkmark) \\
%		\hline
%		Message Queuing (PULL) & \checkmark & \checkmark & \text{P(} \checkmark \text{)} \\
%		\hline
%		Public/Subscribe & \checkmark & \checkmark & \checkmark \\
%		\hline
%		\hline
%	\end{tabular}
%	
%	\caption[MongoDB VoltDB]{
%		\label{tab:mongovolt}
%		\footnotesize{%
%			Comparison of MongoDB and VoltDB.
%		} 
%	}
%\end{table}

\subsection{Comparison with regards to the Siemens case}
Looking at the Siemens case (\cref{sec:SiemensCase}) the new distributed system must act as a single unit, be able to perform park regulations and scale easily with the number of turbines. Furthermore Siemens wish to remove single point of failures. With this in mind, the remote procedure call paradigm is not an option because it is tight coupled and build for a client/server architecture, which is exactly what Siemens is trying to avoid. One could imaging using a partial client/server architecture, with a communication hierarchy, however this would introduce some communication overhead~\cite{Yu1997JavaDSM} and single point of failures to the system.

As mentioned, to perform the park regulation sequence, the windmill farm needs a global shared state, where information about every turbine available to every turbine performing park regulations. Therefore, if the decentralized version of the Siemens case were to be implemented using raw message passing, it would result in building some kind of DSM and/or publish/subscribe abstraction, with roughly the same communication overhead. To save the trouble of developing this abstraction, we conclude that message passing is not an option as communication paradigm.

This leaves DSM and publish/subscribe as remaining paradigms, where one could imagine the Siemens case being implemented using either of the two. The abstraction of shared memory provided by DSM is attractive when looking looking at the Siemens case. As a system developer, being able to think of the global shared state as local memory is to prefer over event handling. The Data Distribution Service defined by the Object Management Group provides a middleware based on Real Time Publish-Subscribe with the ability to maintain a history of messages. This enables the use of publish/subscribe as well as the ability to maintain a global shared state through the message history.

%This leaves DSM and publish/subscribe as remaining paradigms, where one could imagine the Siemens case being implemented using either of the two. The abstraction of shared memory provided by DSM is attractive when looking looking at the Siemens case. As a system developer, being able to think of the global shared state as local memory is to prefer over event handling. Therefore DSM is chosen over publish/subscribe since publish/subscribe introduces event handling to the system.

% The abstraction provided by DSM is attractive when looking looking at the Siemens case. As a system developer, being able to think of the global shared state as local memory is to prefer over thinking about communication semantics. However before before choosing, it is relevant to look at the cost of the DSM abstraction.

%\subsection{Message passing or DSM?}
%
%The abstraction provided by DSM is attractive when looking looking at the Siemens case. As a system developer, being able to think of the global shared state as local memory is to prefer over thinking about communication semantics. However before before choosing, it is relevant to look at the cost of the DSM abstraction.

%Comparing DSM with message passing in terms of processing time and network communication time is not entirely fair since DSM is an abstraction built using message passing. Therefore, Honghui~\cite{lu1995message} argues that it is hard for DSM to outperform message passing, in terms of application execution time, given the larger software-overhead. Honghui has studied and compared a DSM system with message passing system, with the goal to assess the differences in application development time and program execution time between DSM and message passing, and determine the causes of the lower program execution time of DSM systems. He ported 12 different parallel program scenarios to a DSM system called TreadMarks and a message passing system called PVM. For 5 of the scenarios, TreadMarks performed within 10\% of PVM. For 6 of the programs the difference were between 10\% - 30\%. For the last scenario, PVM performed twice as well as TreadMarks. 
%
%%He ported 12 different parallel program scenarios to a DSM system called TreadMarks and a message passing system called PVM and compared the two technologies with regards to programmability and performance. He argues that given DSM is an abstraction built on top of message passing, DSM cannot achieve better performance than message passing, given the larger software-overhead. Therefore the goal is to achieve the same performance as message passing using DSM. For 5 of the scenarios, TreadMarks performed within 10\% of PVM. For 6 of the programs the difference were between 10\% - 30\%. For the last scenario, PVM performed twice as well as TreadMarks. 
%
%Honghui argues that the program execution time is dependent of the logical flow of the program scenario. More messages and more data are sent in TreadMarks, explaining the performance differences. He gives the following reasons for the extra communication in TreadMarks:
%
%\begin{itemize}
%	\item Separation of synchronization and data transfer in TreadMarks. 
%	\item Extra messages to request updates for data in the invalidate protocol used in TreadMarks.
%	\item False sharing.
%	\item Diff accumulation for migratory data in TreadMarks.
%\end{itemize} 
%
%%1) Seperation of synchronization
%% Lazy release consistency: Against data races (which may result uin wrong results). Only the next processor that acquires the lock can access x --> only that processor is informormed of the change to x --> reduce message traffic. Ex: Barriers - No processor overites values before all processors have read the value computed in the previous interation.
%
%%2) Extra messages to request updates for data in the invalidate protocol used in TreadMarks
%% Memory page change communicatin. Modified pages are inviladated after an acquire. Later access causes access miss, which in turn causes installation of an up-to-date copy of the page.
%
%%3) False sharing
%% To objects er allokerede i samme memory page og de skrives til samtidig --> force update af page --> overhead
%
%%4) diff accumulation for migratory data in TreadMarks
%% Multiple-writer protocol to allow wrinting on same page at the same time. Uses a diff algorithm to reduce false sharing effects.
%
%Honghui concludes that the performance of a well optimized DSM system is comparable to that of a message passing system. Furthermore, development of systems with complex communication patterns takes a lot less effort using the DSM paradigm.
%
%In contrast to Honghui, Stumm and Zhou~\cite{stumm1990algorithms} argues that applications using DSM can in fact outperform their message passing counterparts, in a few cases. They argue, that this is possible for the following reasons:
%
%\begin{itemize}
% 	\item DSM algorithms typically move data on demand as they are being accessed, which spreads communication load over a longer period of time, allowing for a greater degree of concurrency. If for example a node uses the shared memory more than others, the node does not need to communicate for every write operation made to the shared memory.
% 	\item For DSM algorithms that sends data in large blocks, communication overhead is reduced. 
%\end{itemize} 
%
%Looking at the Siemens case the two major factors for the communication paradigm choice are scalability and availability.
% With that in mind, BLABLA is not an option because of .. 
%
%Message queuing and RMI offers feature which  



%\subsection{State of the art DSM}

%The regulation algorithm is a recursive algorithm, which performs calculations of new turbine set points using the following parameters:

%\begin{itemize}
%	\item Current state of every turbine in the farm.
%	\item Max capacity of each turbine.
%	\item Historic data from previous calculations.
%\end{itemize} 



%In order to remove the communication overhead from the regulation cycle time, information sharing through the chosen paradigm, DSM, needs to to be detached from the regulation cycle. 

%Looking at the global shared state 

% Removing communication overhead from the park regulation cycle involves the following

%DSM is a paradigm well implemented through by many different frameworks, with their own individual and unique implementation.


%Therefore, to determine which technology that is best suited for the Siemens case, the technologies will be compared on the following set of parameters: 

%standalone: Type?
%Distributed shared memory support
%Distributed shared chache support
%Distributed memeory support
%Performance
%Client/server?
%Framework architecture. distributed shared memeory, client/server or not  
%cost

%With DSM chosen as the communication paradigm, this chapter describes, evaluates and discusses different DSM technologies in order to find the technology best suited for Siemens.

%Standalone, avoid overhead


%\subsection{In memory databases}
%Client/server architecture

%\subsection{In-memory data grids}
%Distributed memory. Only shared for replica reasons

%\subsubsection{Oracle Coherence}
%dist. memory --> distributed chache
%java, .net, c++
%http://www.oracle.com/technetwork/middleware/coherence/distributed-caching-100021.html


%\subsubsection{VMware Gemfire}
%Contact sales
%Not open source

%\subsubsection{Gigaspaces XAP}
%Share nothing architecture

%\subsubsection{ScaleOut Software}
%Java, .NET, C/C++ 
%cash
%Share nothing architecture
%http://www.scaleoutsoftware.com/downloads-resources/downloads/

%\subsubsection{NCache}
%expensive
%http://www.alachisoft.com/ncache/index.html

%\subsubsection{Hazelcast}
%java
%opensource
%http://hazelcast.com/products/hazelcast/

%\subsubsection{Distributed shared cache}


%\subsubsection{Infinispan}
%promising

%\subsubsection{DSM in .NET}
%IP multicast

%casusally consistent: Writes that are potentially causally related must be seen by all processes in the same order. Concurrent writes may be seen in a different order on different machines.

%\subsubsection{AppFabric}
%Runs on IIS


\subsection{Data Distribution Service for Real-Time Systems}
The Data Distribution Service for Real-Time Systems hereafter called DDS is an Object Management Group standard defining a middleware for communication between information provideres and information consumers.
The standard is an API specification and an interoperability wire-protocol that defines a data-centric publish-subscribe architecture \cite{pardo2003omg}.
The standard has been implemented by a number of software vendors, the most popular being the RTI Connext DDS implementation.

The DDS standard provides a number of services that a decentralized system like the one this thesis aims to create could profit from.
Services include amongst others:

\begin{itemize}
	\item Automatic node discovery.
	\item A data-centric publish-subscribe approach that allows for distribution of data by topics, decoupling the nodes of the system.
	\item Cross-platform support.
	\item Fine grained control of Quality of Service parameters.
\end{itemize}

DDS relies on the multicast features of the IP protocol which is described below.

\subsubsection{IP Protocol Addressing Methodologies}

\begin{wrapfigure}{r}{0.225\textwidth}
	\vspace{-20pt}
		\tikzstyle{fact}=[rectangle, draw=none, rounded corners=1mm, fill=blue, drop shadow,
	text centered, anchor=north, text=white]
	\tikzstyle{source}=[circle, draw=none, fill=orange, circular drop shadow,
	text centered, anchor=north, text=white]
	\tikzstyle{target}=[circle, draw=none, fill=blue, circular drop shadow,
	text centered, anchor=north, text=white]
	\tikzstyle{other}=[circle, draw=black!50, fill=white, circular drop shadow,
	text centered, anchor=north, text=white]
	%\tikzstyle{comment}=[rectangle, draw=black, fill=black!60, rounded corners, drop shadow,
	%anchor=west, text=white, text width=6.5cm]
	\vspace{-20pt}
	\begin{tikzpicture}[node distance=0.8cm]
		\node (Source) [source] {};
		\node (Target1) [target,right=of Source] {} edge [<-] (Source);
		\node (Target2) [other,above=of Target1] {};
		\node (Target3) [other,below=of Target1] {};
		\node (Target3) [other,above=of Source] {};
		\node (Target3) [other,below=of Source] {};
		
		\begin{pgfonlayer}{background}
			\path (Source.west |- Target2.north)+(-0.4,0.4) node (a) {};
			\path (Target1.east |- Target3.south)+(0.4,-0.4) node (b) {};
			\path [fill=yellow!40,rounded corners, draw=black!50, dashed] (a) rectangle (b); 
		\end{pgfonlayer}
	\end{tikzpicture}
	\caption{Unitcast}
	\label{fig:networkUnicast}
	
	\vspace{10pt}
		\begin{tikzpicture}[node distance=0.8cm]	
		\node (Source) [source] {};
		\node (Target1) [target,right=of Source] {} edge [<-] (Source);
		\node (Target2) [target,above=of Target1] {} edge [<-] (Source);
		\node (Target3) [target,below=of Target1] {} edge [<-] (Source);
		\node (Target3) [target,above=of Source] {} edge [<-] (Source);
		\node (Target3) [target,below=of Source] {} edge [<-] (Source);
		
		\begin{pgfonlayer}{background}
			\path (Source.west |- Target2.north)+(-0.4,0.4) node (a) {};
			\path (Target1.east |- Target3.south)+(0.4,-0.4) node (b) {};
			\path [fill=yellow!40,rounded corners, draw=black!50, dashed] (a) rectangle (b); 
		\end{pgfonlayer}
	\end{tikzpicture}
	\caption{Broadcast}
	\label{fig:networkBroadcast}
			
	\vspace{10pt}
	\begin{tikzpicture}[node distance=0.8cm]	
		\node (Source) [source] {};
		\node (Target1) [target,right=of Source] {} edge [<-] (Source);
		\node (Target2) [target,above=of Target1] {} edge [<-] (Source);
		\node (Target3) [other,below=of Target1] {};
		\node (Target3) [other,above=of Source] {};
		\node (Target3) [target,below=of Source] {} edge [<-] (Source);
		
		\begin{pgfonlayer}{background}
			\path (Source.west |- Target2.north)+(-0.4,0.4) node (a) {};
			\path (Target1.east |- Target3.south)+(0.4,-0.4) node (b) {};
			\path [fill=yellow!40,rounded corners, draw=black!50, dashed] (a) rectangle (b); 
		\end{pgfonlayer}
	\end{tikzpicture}
	\caption{Multicast}
	\label{fig:networkMulticast}
	\vspace{-60pt}
		
	\vspace{-10pt}
\end{wrapfigure}

The IP protocol defines different communication methodologies among these are: Unicast, Broadcast \cite{RFC0919_Broadcast} and Multicast \cite{RFC1112_Multicast_IGMPv1}.
These methodologies each allow for different kinds of communication.

\paragraph{Unicast} is the simplest and most used methodology, it is provides point to point communication and is the basis for most TCP/IP communication.
% most TCP/IP since it is also used in Anycast
As seen in \cref{fig:networkUnicast} the source node only send packet to one node, this kind of communication works well as long as the data being transfered only needs to be sent to one node.
%\paragraph{Anycast} exsists but is irelevant to our project

\paragraph{Broadcast} is the direct opposite of unicast, it provides one to all communication, illustrated in \cref{fig:networkBroadcast}. All in this context mens all machines on the same subnet or locally behind the nearest router.
The source node only needs to send one packet to reach every other node, copying of the packet is done in the network by switches and routers.
There exist a broadcast address for every subnet and a special address witch broadcast to all machines within the local network separated from the outside with a router.
%The broadcast address for a given subnet can be found by doing a bitwise or operation on the local ip address with the binary complement of the subnet mask ($broadcast~address~=~local~ip~|~!subnet~mask$). To broadcast to all subnets behind the nearest router the address $255.255.255$ can be used, other protocols have similar capabilities.
Broadcast does not support TCP/IP and delivers packet in a best effort way, this means that there is no guarantee or feedback of if the packet arrived, all this has to be built on top if needed.
On large networks there can be a lot of broadcast traffic that every node needs to decide if it needs to respond to or discard.
In the new IPv6 broadcast is no longer supported \cite{RFC4291_AddressingIPv6Draft} and is replaced by multicast protocols, this makes it optional to listen to broadcast traffic and possibly increase performance.

\paragraph{Multicast} is a combination of unicast and broadcast and can be configured to be used as either one, however this is not what it is designed for. Multicast is designed to provide one to many communication as illustrated in \cref{fig:networkMulticast}.
Multicasting works using multicast groups, these groups are maintained by routers and layer 2 switches. The way a node is added to a multicast group is by transmitting a packet requesting a group membership using the IGMP protocol. Each multicast group is associated with a multicast ip address, this ip address is used by the transmitting node when a packet is to be transmitted to a multicast group.
Communicating via multicating is efficient and ensures that each node can be configured to handle a minimum number of packets, routers and switches however has a added responsibility of maintaining information of witch multicast address each port is associated with and transmitting packets to all of them.

\subsubsection{DDS entities}
RTI Connext DDS provides a number of entities that must be present in a system in order to enable commmunication.

\begin{itemize}
	\item \textbf{DomainParticipant} \\
		Domains represent logical, isolated, communication networks. Entities existing in different domains are unable to communicate with each other. The DomainParticipant is created with an integer value specifying the domain it belongs to. The DomainParticipants owns the Topics, Publishers and Subscribers in effect making them belong to a specific domain.
	
	\item \textbf{Publisher} \\
		A Publisher owns and manages a number of DataWriters. The Publisher is responsible for the publishing of data provided by the DataWriters. One Publisher can publish data from several DataWriters on several different Topics.
	
	\item \textbf{Subscriber} \\
		A Subscriber owns and manages a number of DataReaders. The Subscriber is responsible for the receipt of published data on behalf of the DataReaders. One Subscriber can receive data for several DataReaders on several different Topics. 
	
	\item \textbf{Topic} \\
		Topics interconnect DataReaders and DataWriters. DataWriters can subscribe to a Topic and receive data each time a DataReader publishes data to the same Topic.
	
	\item \textbf{DataWriter} \\
		A Datawriter is owned by a single Publisher. It is associated with a single Topic and is able to send data to this Topic using the owning Publisher.
	
	\item \textbf{DataReader} \\
		A DataReader is owned by a single Subscriber. It is associated with a single Topic and is able to receive data published to this Topic using the owning DataReader.
\end{itemize}

\begin{figure}
	\centering
	\includegraphics[width=0.7\textwidth,natwidth=610,natheight=642]{DDSentities.png} 
	\captionsetup{format=plain,font=footnotesize,labelfont={bf,defaultCapFont},labelsep=quad,singlelinecheck=no}
	\caption[DDS entities]{
		\label{fig:publishSubscribe} 
		\footnotesize{%
			DDS entites as presented in RTI Connext Core Libraries and Utilities User's Manual \cite{rtiConnextUsersManual}.
		}
	}
\end{figure}

\subsubsection{Node discovery}
Node discovery in DDS is performed using the Simple Discovery Protocol(SDP).
This protocol describes a sequence of messages that are sent between DDS entities in order for them to automatically discover each other.
Node discovery is important in a system like the Siemens Case described in \cref{sec:SiemensCase} because the regulation of a wind farm requires data from all turbines in order to regulate the overall wind farm production correctly.
The protocol is split into two phases:

\begin{itemize}
	\item Simple Participant Discovery Protocol
	\item Simple Endpoint Discovery Protocol
\end{itemize}

The Simple Participant Discovery Protocol(SPDP) phase facilitates the discovery of DomainParticipants across the network.
After this phase has been completed all DomainParticipants in the system has registered the necessary information to enable future communication with each other.

The Simple Endpoint Discovery Protocol(SEDP) phase facilitates the discovery of DataWriters and DataReaders across the network.
After this phase has been completed all DomainParticipants in the system has registered the necessesary information to enable DataWriters and DataReaders to communicate.

\begin{figure}
	\centering
	\includegraphics[width=0.7\textwidth,natwidth=610,natheight=642]{DDSdiscovery.png} 
	\captionsetup{format=plain,font=footnotesize,labelfont={bf,defaultCapFont},labelsep=quad,singlelinecheck=no}
	\caption[DDS discovery phases]{
		\label{fig:publishSubscribe} 
		\footnotesize{%
			DDS discovery phases as presented in RTI Connext Core Libraries and Utilities User's Manual \cite{rtiConnextUsersManual}.
		}
	}
\end{figure}

Since the SEDP registers all the DataWriters and DataReaders in the system the protocol scales poorly \cite{(CFDP)}. The discovery time and number of packets sent grows as the number of endpoints increase. This affects the scalability of the system directly. To avoid the increase in discovery time different protocols has been proposed to solve the scalability problem. 

\paragraph{Content-based Filtering Discovery Protocol(CFDP)} is similar to SDP during the SPDP phase but it differs in the SEDP phase where changes has been made in order to facilitate content-based discovery of endpoints. The filtering allows the DomainParticipant to reduce the number of discovery messages to the number of endpoints that needs to communicate with the DomainParticipant and its local endpoints. Currently CFDP has only been tested using unicast and here the protocol outperforms SDP in terms of network usage, packages sent and discovery time. However if multicast is used as the underlying transport protocol SDP outperforms CFDP. This happens because SDP only needs one multicast for all DomainParticipants. CFDP in contrast will need one multicast address for each content filter used. This limitation in CFDP is proposed as a subject for further research.

Since CFDP does not outperform SDP when using multicast, CFDP is not suited as the discovery protocol for the system described in this thesis.

\paragraph{A discovery protocol based on Bloom filters} has been proposed as well \cite{monedero2009dds}. Similar to CFDP the Bloom filter approach aims to reduce the number of packets that needs to be exchanged in order to discover endpoints in the system. The Bloom filter approach introduces a Bloom filter for each DomainParticipant that contains information describing the local endpoints. During the SPDP phase the Bloom filters are distributed to all DomainParticipants. In this way information of all endpoints in the system has been distributed when the SEDP phase begins allowing the DomainParticipants to decrease the number of discovery messages sent to the number of endpoints that it must know in order to facilitate communication for its local DataReaders and DataWriters. 

Again the main advantage of using Bloom filters is found in the unicast scenario. In a multicast scenario there still are advantages of using the Bloom filter approach because the DomainParticipants are able to limit the number of discovery messages compared to SDP. The advantage of using the Bloom filter approach increases as the ratio between endpoints and DomainParticipants increase.

Since a wind farm has a limited number of endpoints, less than 300 turbines, the decrease in discovery time by using the Bloom filter approach would be minor. Therefore the Bloom filter approach is not further investigated in this thesis. \newline

The Simple Discovery Protocol does not scale well with the number of endpoints. This problem is currently under research and solutions have been proposed. As of the time of writing none of the proposed solutions to the problem are well suited for use in this thesis. Further research in the area may present new solutions to the problem that are well suited for use in a multicast enabled system with less than 300 nodes.

\subsubsection{Quality of Service}
DDS provides Quality of Service(QoS) parameters to enable very fine grained control over the behavior of the deployed DDS entities in a system. To create a decentralized system several of the QoS parameters are interesting.

\begin{itemize}
	\item \textbf{Multicast} \\
		Enabling multicast in the system will decrease the network traffic in the system, which in turn increases scalability.
		
	\item \textbf{History} \\
		The History parameter will enable the system to keep a log of messages received during operation thus eliminating the need for another storage option. This saves the system from using additional resources for storing and retrieving values from a file or a local database.
		
	\item \textbf{Lifespan} \\
		The History parameter enables the system to keep a log of messages but the history may only be relevant for a specified amount of time. In order not to keep irrelevant messages it can be specified how long they are relevant.
	
	\item \textbf{Durability} \\
		To make sure a new turbine can start operation the Durability parameter can be used in order to update the new turbine with the history of messages in the system. In this way the turbine does not need to wait to receive new data from individual turbines to start regulation, instead it can start regulation as soon as the history is received.
		
	\item \textbf{Reliability} \\
		This parameter regulates the reliable delivery of messages in the system, which affect the operation of the turbines and their regulation.
		
	\item \textbf{Liveness} \\
		The Liveness parameter specifies how liveness of a datawriter is determined. This allows the system to detect when a turbine is no longer available and to respond in an appropriate way.
\end{itemize}

Using the QoS of DDS a system can be configured to fit its environment and its requirements very tightly allowing efficient and reliable operation.

\subsection{Conclusion}
The RTI Connext Data Distribution Service for Real Time Systems enables both communication using publish/subscribe as well as the ability to maintain a global shared state by preserving message history. Coupled with the fact that the middleware provides automatic node discovery, cross-platform support and fine grained control using Quality of Service parameters the RTI Connext Data Distribution Service for Real Time Systems is the best choice of middleware for the Siemens Case (\cref{sec:SiemensCase}).