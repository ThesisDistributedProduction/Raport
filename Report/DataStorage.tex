\chapter{Data storage}
Data storage is important in a system like a windmill farm.
A lot of data must be persisted like weather data, health data of the different parts of every windmill and production data.
These records are important both for immediate use to view the current state of the system but also for review in the future for instance to predict weather trends or replace worn down parts of windmill before they break completely.

Currently data is aggregated from each windmill and stored on a central node.
This node will over time aggregate hundreds of gigabytes of information.
The data on the node is secured by backup but it is still a single point of failure.
Take out the data storage node or the communication to it and a lot of information will be lost.

By distributing the data of the system between all the connected nodes we achieve better redundancy because the data is present on many different nodes.
Should a node become unavailable another node can communicate the same data in effect strengthening the availability of the system.

This chapter contains a description of a number of relevant storage technologies and a discussion of which technology is the best suited for a system like the Siemens case.

\section{Relational storage, SQL}
The traditional way of storing data is in a Relational Database Management System(RDBMS).
These databases rely on a schema to arrange data in entities and their relations.
Using SQL it is easy to query data and to do aggregate operations.
RDBMSs support ACID transactions which ensures operations in the database are processed reliably.

A shortcoming of the RDBMSs is the problem with object-relational mapping also known as the Impedance Mismatch Problem\cite{Fowler:IntroNoSQL, Neward:TheVietnamOfComputerScience}.
The relational structure of the RDBMSs does not map well to the object-oriented structure the most popular programming languages encourage.
Often an object is an aggregate of a number of attributes.
In the context of the object-oriented program the object is seen as one entity.
In the context of the RDBMS the attributes of the object-oriented object is often scattered between multiple tables in the database to ensure consistency and avoid duplicate data.
This mismatch between object representation and relational representation can cause both performance problems, the JOIN operation in SQL is very costly, as well as considerable development time spent mapping one structure to the other.
The performance problem multiplies in a distributed database if the RDBMS must do JOIN operations across the network in order to aggregate data.

Another problem with a traditional RDBMS is that they are designed for vertical scaling\cite{Atzeni:TheRelationalModelIsDead}. If a traditional RDBMS has problems handling data the solution is to add a bigger harddrive or invest in a faster CPU. This makes sense in a world were hardware is very expensive like it was when the traditional RDBMSs saw the light of day\cite{Stonebraker:TheEndOfAnArchitecturalEra}. Today horizontal scaling is preferred. If a system has a problem with the data load add another machine or add five others if that is what it takes.

\section{Schema-less storage, NoSQL}
Since 2009 the schema-less storage methods have become increasingly popular.
Relational databases could no longer keep up with the task of storing and querying big data.
A new breed of schema-less storage systems became popular because they could handle some of the problems big data caused for the relational storage systems.

Essentially the schema-less storage systems, or NoSQL databases, are databases without the relational schema constraint imposed by the relational databases. The schema-less databases can be divided roughly into four groups\cite{Fowler:IntroNoSQL}:

\begin{itemize}
\item Document
\item Key-value
\item Column-family
\item Graph
\end{itemize}

\section{Relational storage that scale, NewSQL}

\section{SQL, NoSQL or NewSQL?}

\section{Schema-less databases}

\section{Conclusion}