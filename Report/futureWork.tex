\chapter{Future work}
\label{sec:futWork}
During the work with this thesis we found the following areas which could be interesting for further research:

\section{Full scale experiment}
The current test setup is limited in the number of turbines that can be simultaneously simulated. This has prevented exploring the true limits of a decentralized solution.
A test setup with 1000 or more turbines would press the decentralized solution beyond the current limits of the centralized solution, thereby proving that a decentralized approach scales significantly better than the centralized. Currently Siemens Wind Power has wind farms typically are not this large\cite{simensOnShoreProjects,simensOffShoreProjects}. 


\section{Identify the hardware limitation of the distributed solution's scalability}
In the experiments it is seen how the decentralized solution scales in terms of cycle times, however the hardware performance parameters are not part of the experiments.
Parameters like memory consumption, network throughput and processing time used, if these parameters are plotted against the number of turbines, a trend might emerge that could tell how many turbines the system could support with for example a system with a gigabit network adapter.


\section{Implement missing system components}
During the research phase a number of components where identified which would need to be used in a fully distributed version of the current Siemens system.
A lot of complications could arise when build such a system. The identified components needs to tested and verified to work together and that they fulfill the requirements of a decentralized solutions. 


\section{Clustered decentralized solution}
The decentralized solution is not expected to scale infinitely, every time a turbine is added to the system we also add more load to the cpu and network adapter.
Therefore a system where nodes work together in smaller groups called clusters, would allow limiting the number of nodes witch are included in the cluster's regulation cycle. 
The clusters would only communicate together sparingly in order for the system to achieve a balanced production across clusters.
The clusters must be designed to be large enough to be stable in the event of a node failure.
This should push the scalability limit much further.


