\chapter{Future work}
\label{sec:futWork}
During the work with this thesis we found the following areas which could be interesting for further research:

\section{Full scale experiment}
The current test setup is limited in the number of turbines that can be simultaneously simulated. This has prevented exploring the true limits of a decentralized solution.
A test setup with 1000 or more turbines would press the decentralized solution beyond the current limits of the centralized solution, thereby proving or disproving if a decentralized approach scales significantly better than a centralized. The current Siemens Wind Power wind farms are typically not this large~\cite{simensOnShoreProjects,simensOffShoreProjects}.


\section{Identify the hardware limitation of the distributed solution's scalability}
In the experiments it is seen how the decentralized solution scales in terms of cycle times, however the hardware performance parameters are not part of the experiments.
Parameters like memory consumption, network throughput and processing time. These parameters could be plotted against the number of turbines, a trend might emerge that would identify how many turbines the system could support with a given system setup.


\section{Implement missing system components}
During the research phase a number of components where identified which are needed in a fully distributed version of the current Siemens system.
However many complications could arise when building such a system. The identified components needs to be tested and verified that they work together and that they fulfill the requirements of a decentralized solution. 


\section{Clustered decentralized solution}
The decentralized solution is not expected to scale infinitely, every time a turbine is added to the system we also add more load to the CPU and network adapter.
Therefore a system where nodes work together in smaller groups called clusters, would allow limiting the number of nodes which are included in the regulation cycle of the cluster and thereby also the required resources needed. 
The clusters would only communicate together in order for the system to achieve a balanced production across clusters, however it is thought to be possible to achieve with a much lower communication between clusters compared to individual nodes.
This should make for a system that is far more scalable.


