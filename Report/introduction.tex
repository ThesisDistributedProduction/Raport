\chapter{Introduction}

\section{Thesis motivation}
Today windmills in a windmill farm are connected to a single server that aggregates data, perform calculations, store data and handle communication with the outside world. 
These servers do not scale well with the size of the windmill farm, and they are a single point of failure. 
Therefor Siemens wishes to remove the servers by utilizing free capacity of the computers already residing in every windmill. 
Currently there is some limited redundancy in data and availability but this could be greatly improved by distributing data and communication to the windmills. 
Ease of access must be maintained even though computation and data is distributed. 
This means routing traffic to a windmill with free capacity through. 
This means routing traffic to a windmill with free capacity through  a single interface.

\section{Thesis aim}

\begin{itemize}
	\item How do we distribute a database and computation across a production environment in the best possible way?
	\item How do we define and measure performance?
	\item Can it provide data redundancy and outperform current systems?
	

\end{itemize}

We aim to investigate the possibility of making a framework and associated tools for developing a distributed system. 
This framework must be able to handle computation distributed on several nodes, communication between those nodes and distribution of data. 
The communication can be built on top of existing standards as for instance DDS. Data distribution can be built using existing systems like MongoDB. 
Distribution of computation tasks is the main research area and will be the focus of this thesis.

In order to achieve distributed computation on several nodes the framework must be able to perform load balancing and control the distribution of tasks on the nodes in the system. 
Furthermore the framework must have a single interface for control of, and interaction with, all the nodes.  
The goal is to create a test system, that can distribute and perform tasks but also to be able to plan ahead of time and know if there is available computation time.

The case from Siemens Windpower is an example of a production environment where the framework could be utilized. 
Our goal is not to make a framework that is only  specific for this case but to make a general framework for this and similar cases.

\section{Approach}

\section{Outline}
The remainder of this thesis is organized into the following chapters...

\section{Audience}
This thesis is aimed at an audience with a basic knowledge of...
