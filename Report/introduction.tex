% !TeX spellcheck = en_US
\chapter{Introduction}

% Hvorfor starter ved med at forklare DIstributed systems burde vi ikke starte med problem stillingen og derefeter forklare at vi tager tilgangen til det med distributed system og til dels peer to Peer network.

A distributed system is a network of hardware or software components, which communicates and coordinate their actions only by message passing\cite{coulouris2005distributed}, as illustrated on \cref{fig:distributedSystem}. Each component have their own local memory every component and interact with each other in order to achieve a common goal. The nodes can be physically close, connected via a local network, or geographically distant, connected by a wide area network. An important goal of a distributed system is location transparency\cite{coulouris2005distributed} to create the illusion of the entire system acting as a single computer even though it is comprised of several nodes. Examples of distributed systems vary from aircraft control systems to the Internet to massively multiplayer online games. 
% Er det bare mig eller er det lidt rigeligt at sige at systemet agere som en enkelt computer? Lad hellere sige noget om at opnå et fælles mål? Multiplayer spil er vel ikke distribuerede systemer, men mere server client systemer.

\begin{figure}
	\centering
	\includegraphics[width=0.7\textwidth,natwidth=510,natheight=542]{DistributedSystem.jpg} 
	\captionsetup{format=plain,font=footnotesize,labelfont={bf,defaultCapFont},labelsep=quad,singlelinecheck=no}
	\caption[Distributed Computing System with 2 nodes]{
		\label{fig:distributedSystem} 
		\footnotesize{%
			A distributed system with 4 nodes.
		}
	}
\end{figure}

Distributed systems offer many benefits over centralized systems including the following\cite{IBM2005TXSeries}:
\begin{itemize}
	\item Scalability: It is easy to add nodes to the system, should the size of the system increase.
	\item Redundancy: Several nodes can provide the same service, so if a node crashes, there are many to replace it. Additionally, from a cost perspective, each node does not have to be expensive, because many smaller nodes can be used as replacement.
\end{itemize}
% Vi burde nok også snakke om ulemper? Og det er vel bare ting som er mulige at opnå?

\section{Siemens Wind Power case}

% Beskriv topologi

\label{sec:SiemensCase}
Siemens Wind Power is among the leading wind turbine manufacturers in the world. Siemens builds wind farms of different sizes ranging form single turbines to well above one hundred turbines. \cite{simensOffShoreProjects, simensOnShoreProjects}.

Today turbines in a wind farm at Siemens are equipped with computers for the purpose of regulating power production parameters, data collection and for communication with the rest of the system. The current setup is illustrated on \cref{fig:currentSiemensSetup}. Every turbine is connected to a Wind Power Supervisor, which is a central component that aggregates data, perform calculations, store data and handle wind farm communication with the outside world. At Siemens up to 8  Wind Power Supervisors are present pr. wind farm.

For every transformer station there is a High Performance Park Pilot from here on refereed to as Park Pilot. The Park Pilots job is to monitor all turbines performance parameters and issue correct set points for power production.

A database is deployed on every  Wind Power Supervisor, for aggregated data collected from each turbine. Every turbine has a database as well for data logging purposes. The turbines, Wind Power Supervisors and Park Pilots are connected with a gigabit network, which currently has plenty of extra capacity. The system handles more than 50 control points and 200 measurement points and samples these every 50 ms. The Park Pilots the power production when needed. See \cref{fig:dataComputationSequence} for the regulation sequence.

\begin{figure}
	\centering
	\includegraphics[width=0.7\textwidth,natwidth=610,natheight=642]{SystemOverviews.png} 
	\captionsetup{format=plain,font=footnotesize,labelfont={bf,defaultCapFont},labelsep=quad,singlelinecheck=no}
	\caption[The current Siemens wind farm system overview]{
		\label{fig:currentSiemensSetup} 
		\footnotesize{%
			The current Siemens wind farm system overview.
		}
	}
\end{figure}

%\begin{figure}
%	\centering
%	\includegraphics[width=0.7\textwidth,natwidth=610,natheight=642]{SystemOverviewsFuture.png} 
%	\captionsetup{format=plain,font=footnotesize,labelfont={bf,red},labelsep=quad,singlelinecheck=no}
%	\caption[Illustrates the future Siemens windmill farm setup]{
%		\label{fig:futureSiemensSetup} 
%		\footnotesize{%
%			This figure illustrates the future Siemens windmill farm setup.
%		}
%	}
%\end{figure}

\begin{figure}
	\centering
	\begin{sequencediagram} %Created using pgf-umlsd
		\newthread{reg}{:Park Pilot}
		\newinst[2]{turbine}{:Turbine}
	
		\begin{sdblock}{each turbine}{}
			\mess[1]{reg}{getCurrentStatus}{turbine}
			\mess[1]{turbine}{status}{reg}
		\end{sdblock}
		
		\begin{call}{reg}{calculateAllSetpoints()}{reg}{}
		\end{call}
	
		\begin{sdblock}{each turbine}{}
			\mess[1]{reg}{setNewSetpoint}{turbine}
		\end{sdblock}
					
	\end{sequencediagram}

	\captionsetup{format=plain,font=footnotesize,labelfont={bf,defaultCapFont},labelsep=quad,singlelinecheck=no}
	\caption[Regulator calculation sequence]{
		\label{fig:dataComputationSequence} 
		\footnotesize{%
			Regulator calculation sequence.
		}
	}
\end{figure}

\section{Thesis motivation}

Todays setup at Siemens Wind Power (\cref{sec:SiemensCase}) is an example of a system wished to be made less centralized. The current setup poses the following problems to Siemens:  

\begin{itemize} 
	\item Single point of failure. Should a  Wind Power Supervisor fail, a part of the windmill farm will become unavailable.
	\item Low scalability. The  Wind Power Supervisors and Park Pilots does not scale with the number of windmills. This is also reflected in the park regulation performance. Every turbine needs to return their current status to the Park Pilot, the set points for each turbine is calculated in the Park Pilots and finally the new set points can be pushed to the turbines.
\end{itemize}

Siemens wishes to decentralize the Wind Power Supervisor and Park Pilots by distributing their functionality to the turbines, utilizing the free capacity of the computers already residing in every turbines. This would increase redundancy and scalability and lower the possibility of a single point of failure. Furthermore, Siemens would like to look for ways to optimize the Park Pilots regulation sequence (see \cref{fig:dataComputationSequence}).


\section{Problem statement}

The purpose of this thesis is to design, implement and evaluate a distributed system solution for the Siemens Wind Power case. The solution must be able to provide the same features as the current solution. The goal of the new solution is to improve scalability, availability and performance by distributing the Park Pilot and the Wind Power Supervisor onto the turbines and thereby eliminate single point of failures and make performance scale with the amount of turbines within the wind farm. 

To realize this the system must be redesigned as a distributed system. The solution must be able to handle external requests, communication between the nodes and distribution of data, according to the Siemens case. Furthermore the solution must have a single interface for control of, and interaction with, all the nodes, in order to maintain the illusion of the wind farm serving as a single system. This means ease of access must be maintained even though computation and data is distributed among nodes. Traffic must be routed to a turbine controllers with free capacity through a single interface, without external systems being aware of it.

This roughly leaves three essential components: A component that handles distribution of data, a component that handles communication between nodes and a load balancer, to keep track of available resources on each node. The aim to investigate, analyze and evaluate state of the art technologies within each component area and choose the technologies best suited for the Siemens case. The technologies will be weighed in terms of features with regards to the Siemens case and performance.

Furthermore we aim to develop a prototype, which runs the developed solution, for proof of concept purposes. The prototype will be compared to the existing Siemens solution with respect to performance, availability and redundancy. ASK SIEMENS HOW TO COMPARE AND MORE CONSTRAINTS??? YOLO 

The Siemens case presents the following constraints to the project:
\begin{itemize}
	\item CPU power. Our solution must be able to run on a standard consumer hardware.
	\item Network bandwidth. Siemens uses gigabit network.
	\item Topology. Siemens wind farms uses ring and star topology.
\end{itemize}

