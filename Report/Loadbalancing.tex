% !TeX spellcheck = en_GB
\chapter{Load balancing}

When dealing with redundant distributed systems, there often exists more than one node capable of doing some work.
However one node is not able to handle the workload in a satisfying manner.
This means the nodes must cooperate and balance out the workload.
This is done using a load balancer with a node balancing algorithm.

In this solution the load balancer needs to balance external connections to different protocols like HTTP and Modbus, however a solution witch can be extended to any restful protocol is needed. Also balancing of node roles depending on the amount incoming traffic on different interfaces will be needed.

Load balancers can also provide different features like bundling requests, security, discovering bad nodes and caching (Squid). This can offload the servers behind.

The following requirements to the system exists:
\begin{description}
	\item Robustness
	\item Protocol flexible
	\item support TCP Handoffs (for non restful applications)
	\item Must be a distributed component
\end{description}

\section{Levels of balancing}
\begin{description}
	\item[OSI 3] Network/IP
	\item[OSI 4] Network/IP
	\item[OSI 7] {Application level, like http balancing, allows balancing strategies based on url and user location.}
\end{description}

What we would like is a transport layer protocol.
\cite{Ludwig:SwarmIntelligenceGridLoadBalancing} Implements a particle swam based algorithm, and discuses quality parameters.

\section{Existing solutions}
\begin{description}
	\item[Linux Virtual Server: IPVS] Is implemented in the linux kernal version 2.4 and 2.6. Works at the IP level. Useed byt big sites sourceforge.net, layer 3.
	\item[Google Compute Engine: Load Balancer]: Proprietary. layer 3 and 7.
\end{description}