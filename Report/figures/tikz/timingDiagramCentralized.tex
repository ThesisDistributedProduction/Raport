

{ %The brackets issolate the enviroment

\tikzstyle{line}		 	= [draw]

\makeatletter
\ifcsname c@wavenum\endcsname %Only create one counter
\else
	\newcounter{wavenum}
\fi
\makeatother

\newcommand*{\bitvector}[3]{
  \draw[fill=#3] (t_cur) -- ++( .1, .3) -- ++(#2-.2,0) -- ++(.1, -.3)
                         -- ++(-.1,-.3) -- ++(.2-#2,0) -- cycle;
  \path (t_cur) -- node[anchor=mid](textNode) {#1} ++(#2,0) node[time] (t_cur) {};
  }

% \known{val}{length}
\newcommand*{\known}[2]{
    \bitvector{#1}{#2}{white}
}

% \unknown{length}
\newcommand*{\unknown}[2]{
    \bitvector{#1}{#2}{black!20}
}

% \nextwave{name}
\newcommand{\nextwave}[1]{
  \path (0,\value{wavenum}) node[time] (t_cur) {};
  % \path (0,\value{wavenum}) node[left] {#1} node[time] (t_cur) {};
  \addtocounter{wavenum}{-1}
}

\newcommand{\timeSpanLabel}{
	\node (CycleTimeLabel) [rectangle, above = 0.7cm of textNode, inner sep=0pt] {~~~~~~~CycleTime~~~~~~~};	  
}

\newcommand{\timeSpanA}{
	\node (t_timeSpanA) [point, above = 0 of t_cur] {};	  
}

\newcommand{\timeSpanB}{
	\node (t_timeSpanB) [point, above =0 of t_cur] {};

  \graph[use existing nodes]{
  	t_timeSpanA --[time span=1cm] CycleTimeLabel;
   	CycleTimeLabel.south --[time span=-0.24cm] t_timeSpanB;
  }; 
    	
}


%%% End of timing.sty
\begin{tikzpicture}[
	point/.style={inner sep=0pt}, %circle,minimum size=2pt,fill=red},
	draw=black, 
	yscale=.8,
	xscale=1,
	hv path/.style={to path={-| (\tikztotarget)}},
	vh path/.style={to path={|- (\tikztotarget)}},
	skip loop v/.style={to path={-- ++(#1,0) |- (\tikztotarget)}},		
	skip loop h/.style={to path={-- ++(0,#1) -| (\tikztotarget)}},
	time span/.style={to path={-- ++(0,#1) -| (\tikztotarget)}},
	graphs/every graph/.style={edges=rounded corners}	
	]
	
  \tikzstyle{time}=[coordinate]
  \setlength{\unitlength}{1cm}
  \setcounter{wavenum}{0}
    
  %\nextwave{Regulation Time} \unknown{SendData}{2} \known{WaitForData}{5} \unknown{ReciveData}{2} \unknown{Calculate}{2}\unknown{SendSP}{2}
  \nextwave{} \timeSpanA \unknown{ReqData}{2} \known{WaitForData}{3} \unknown{ReadData}{2} \timeSpanLabel \unknown{Calculate}{2}\unknown{Send SP}{2} \timeSpanB \known{Buffer}{3}
  
  
  
\end{tikzpicture}
}
