

{ %The brackets issolate the enviroment

\makeatletter
\ifcsname c@wavenum\endcsname %Only create one counter
\else
	\newcounter{wavenum}
\fi
\makeatother

\newcommand*{\bitvector}[3]{
  \draw[fill=#3] (t_cur) -- ++( .1, .3) -- ++(#2-.2,0) -- ++(.1, -.3)
                         -- ++(-.1,-.3) -- ++(.2-#2,0) -- cycle;
  \path (t_cur) -- node[anchor=mid] {#1} ++(#2,0) node[time] (t_cur) {};
}

% \known{val}{length}
\newcommand*{\known}[2]{
    \bitvector{#1}{#2}{white}
}

% \unknown{length}
\newcommand*{\unknown}[2]{
    \bitvector{#1}{#2}{black!20}
}

% \nextwave{name}
\newcommand{\nextwave}[1]{
  \path (0,\value{wavenum}) node[left] {#1} node[time] (t_cur) {};
  \addtocounter{wavenum}{-1}
}

% \begin{wave}[clkname]{num_waves}{clock_cycles}
\newenvironment{wave}{
  \begin{tikzpicture}[draw=black, yscale=.8,xscale=1]
    \tikzstyle{time}=[coordinate]
    \setlength{\unitlength}{1cm}
    \setcounter{wavenum}{0}
    
}{\end{tikzpicture}}

%%% End of timing.sty

\begin{wave}
 \nextwave{Regulation Time} \unknown{SendData}{2} \known{WaitForData}{5} \unknown{ReciveData}{2} \unknown{Calculate}{2}
\end{wave}
}
