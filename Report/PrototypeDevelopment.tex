% !TeX spellcheck = en_US
\chapter{Prototype development}
The prototype in this project will consist of X virtual machines.



\section{Platform}

Windows
Pro
\begin{itemize}
	\item Well known to most people (easy to get going) 
	\item 
\end{itemize}

Bad
\begin{itemize}
	\item expensive
\end{itemize}



Linux
Pro
\begin{itemize}
	\item Market leader [?] Need Source
	\item Linux Virtual server
	\item Linux Containers
	\item Open source / free
	\item More configurable with respect to scheduler (non preemptive (default), preemptive voluntary, RT ) 
	\begin{itemize}
		\item \url{http://www.linuxtopia.org/online_books/linux_kernel/kernel_configuration/re152.html}
		\item \url{http://stackoverflow.com/questions/5174955/what-is-voluntary-preemption}
		\item \url{http://lwn.net/Articles/146861/}
		\item \url{https://www.osadl.org/uploads/media/ECE-2011-09.pdf}
		\item \url{http://www.linux.com/news/featured-blogs/200-libby-clark/710319-intro-to-real-time-linux-for-embedded-developers}
								
	\end{itemize}
\end{itemize}



%user: shared
%password: windfarm

%https://forums.virtualbox.org/viewtopic.php?f=6&t=63556&start=165
 

The prototype is split int the following 3 parts:
\begin{itemize}
	\item DataSimulator
	\item HPPP component
	\item Monitor (UI component)
	\end{itemize}


\begin{tikzpicture} 
\begin{umlseqdiag} 
	\umldatabase{DB} 
	\umlmulti{DataSimulator}
	\umlmulti{Regulation Controller}
	\umlmulti{Monitor}
	\begin{umlcall}[op=getAlldata(), type=asynchron, return=0]{DB}{DataSimulator}
		\begin{umlfragment}[type=ForAll, label=Samples, inner xsep=8, fill=white!10]
			\begin{umlcall}[op=Simumlation data, type=synchron, return=setpoint]{DataSimulator}{Regulation Controller}
			
			\end{umlcall}
			\begin{umlfragment}[type=opt, fill=white!10]
				\begin{umlcall}[op=UpdateUI(), type=asynchron, return=0]{DataSimulator}{Monitor}
				\end{umlcall}
			\end{umlfragment}
		\end{umlfragment}


	\begin{umlcall}[op=saveData(), type=asynchron, return=0]{DataSimulator}{DB}
	\end{umlcall}	
	\end{umlcall}
	
	
\end{umlseqdiag} 
\end{tikzpicture}

